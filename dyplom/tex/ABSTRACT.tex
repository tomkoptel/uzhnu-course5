\pagenumbering{gobble}

\textbf{ПІБ}: Коптель Артем Олегович

\textbf{Назва}: React Native vs Flutter vs Kotlin Native. Порівняння кросс-платформених фрейморків для розробки мобільних додатків

\textbf{Факультет}: інформаційних технологій

\textbf{Спеціальність}: 122 ``Комп'ютерні науки''

\textbf{Науковий керівник}: доц. О.В. Міца

По даній роботі опубліковано $0$ робіт.

\begin{center}
\textbf{АНОТАЦІЯ}
\end{center}
%\chapter*{\Large \vspace*{-20mm}$$\mbox{АНОТАЦІЯ}$$}
%\addcontentsline{toc}{chapter}{Анотація} \thispagestyle{headings}

Дана робота присвячена дослідженню фреймворків розробки мобільних додатків.

Дана магістерська робота містить 3 розділи, 10 підрозділів.

{\bf Ключові слова:} мобільна розробка, Flutter, React Native, KMM

\newpage

\textbf{Name}: Artem Koptel

\textbf{Title}: React Native vs Flutter vs Kotlin Native. Comparison of crossplatform frameworks for the mobile development.

\textbf{Faculty}: Faculty of Information Technologies

\textbf{Speciality}: 122, Computer sciences

\textbf{Supervisor}: Dr. Alexander Mitsa. 

$0$ papers were submitted for publication on this topic.

\begin{center}
\textbf{ABSTRACT}
\end{center}
%\chapter*{\Large \vspace*{-20mm}$$\mbox{АНОТАЦІЯ}$$}
%\addcontentsline{toc}{chapter}{Анотація} \thispagestyle{headings}

The following work is dedicated exploration of the crossplatform frameworks for the mobile development.

Thesis work contains 3 chapters, 10 sections.

{\bf Keywords:} finite field, group, computer algebra, group algebra, random methods.

