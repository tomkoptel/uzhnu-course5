\chapter{React Native}
\label{ch1}

\section{Декларативний}
\label{section.2.1}

React спрощує створення інтерактивних інтерфейсів.
Вам потрібно лише описати, як різні частини інтерфейсу виглядають у кожному стані вашого додатку і React ефективно оновить та відрендерить лише потрібні компоненти, коли ваші дані зміняться.
Декларативні інтерфейси роблять наш код більш передбачуваним і його набагато легше налагоджувати.

\section{Заснований на компонентах}
\label{section.2.2}
React Native заснований на основі інкапсульованих компонентів, які керують власним станом, і з них будують складні інтерфейси.
Оскільки логіка компонентів написана на JavaScript, замість шаблонів, ми з легкістю можеме передавати складні дані у додатку і зберігати стан окремо від DOM.

\section{JSX}
\label{section.2.3}
Розглянемо оголошення змінної:
\begin{lstlisting}[style=light, language=Python,label={lst:jsx_hello},caption=JSX Hello World]
const element = <h1>Hello, world!</h1>;
\end{lstlisting}

Цей кумедний синтаксис тегів не є ні рядком, ні HTML.

Він має назву JSX, і це розширення синтаксису для JavaScript.
Його використовує React, щоб описати інтерфейс користувача.
JSX може нагадувати мову шаблонів, але з усіма перевагами JavaScript.

React використовує той факт, що логіка виводу пов’язана з іншою логікою інтерфейсу користувача: як обробляються події, як змінюється стан з часом і як дані готуються для рендерингу.

Замість того, щоб штучно відокремити технології, розмістивши розмітку і логіку в окремих файлах, React розділяє відповідальність між вільно зв’язаними одиницями, що містять обидві технології і називаються “компонентами”.
React не вимагає використання JSX, але більшість людей цінують його за візуальну допомогу при роботі з інтерфейсом користувача в коді JavaScript.
Він також дозволяє React показати зрозуміліші повідомлення про помилки та попередження.

\section{Система побудування Expo}
\label{section.2.4}
React Native пропонує два різних способи побудування проекту \textbf{«керований»} та \textbf{«простий»} робочими процесами.
\begin{itemize}
    \begin{item}
        За допомогою \textbf{керованого} робочого процесу ви пишете лише інструменти та служби JavaScript / TypeScript та Expo, які піклуються про все інше.
    \end{item}
    \begin{item}
        У \textbf{простому} робочому процесі ви маєте повний контроль над кожним аспектом власного проекту, а інструменти та послуги Expo трохи більш обмежені.
    \end{item}
\end{itemize}

В обох випадках використовується екосистема інструментів, що спрощують процес пободування проекту, ховаючи деталі та називається Expo.
Expo - платформа для універсальних додатків React, набір інструментів та служб, побудованих навколо React Native та власних платформ, які допомагають вам розробляти, будувати, розгортати та швидко переглядати iOS, Android та веб-додатки з тієї самої кодової бази JavaScript / TypeScript.\cite{expo_doc}
