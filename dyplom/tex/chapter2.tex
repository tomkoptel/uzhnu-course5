\chapter{React Native}
\label{ch1}

\section{Декларативний}
\label{section.2.1}

React спрощує створення інтерактивних інтерфейсів.
Вам потрібно лише описати, як різні частини інтерфейсу виглядають у кожному стані вашого додатку і React ефективно оновить та відрендерить лише потрібні компоненти, коли ваші дані зміняться.
Декларативні інтерфейси роблять ваш код більш передбачуваним і його набагато легше налагоджувати.

\section{Заснований на компонентах}
\label{section.2.2}
Створюйте інкапсульовані компоненти, які керують власним станом, а з них будуйте складні інтерфейси.
Оскільки логіка компонентів написана на JavaScript, замість шаблонів, ви з легкістю можете передавати складні дані у вашому додатку і зберігати стан окремо від DOM.

\section{Вивчіть лише раз — пишіть будь-де}
\label{section.2.3}
Ми не робимо припущень щодо стеку технологій які ви використовуєте, тому ви можете розробляти нові функції в React, не переписуючи існуючий код.
React також може рендеритись на сервері, використовуючи Node, і приводити в дію мобільні додатки, які використовують React Native.
