\newpage
\renewcommand{\proofname}{Доведення}
\renewcommand{\chaptername}{РОЗДІЛ}
\chapter[Текст для содержания]{Практична реалізація}
\label{ch3}

\lstset{ %
language=GAP,                 % выбор языка для подсветки (здесь это С)
basicstyle=\small\sffamily\linespread{0.8}, % размер и начертание шрифта для подсветки кода
%numbers=left,               % где поставить нумерацию строк (слева\справа)
%numberstyle=\tiny,           % размер шрифта для номеров строк
%stepnumber=1,                   % размер шага между двумя номерами строк
%numbersep=5pt,                % как далеко отстоят номера строк от подсвечиваемого кода
backgroundcolor=\color{white}, % цвет фона подсветки - используем \usepackage{color}
showspaces=false,            % показывать или нет пробелы специальными отступами
showstringspaces=false,      % показывать или нет пробелы в строках
showtabs=false,             % показывать или нет табуляцию в строках
frame=single,              % рисовать рамку вокруг кода
tabsize=2,                 % размер табуляции по умолчанию равен 2 пробелам
captionpos=t,              % позиция заголовка вверху [t] или внизу [b] 
breaklines=true,           % автоматически переносить строки (да\нет)
breakatwhitespace=false, % переносить строки только если есть пробел
escapeinside={\%*}{*)}   % если нужно добавить комментарии в коде
}

\section{Вибір середовища програмування}
\label{section.3.1}

Для програмної реалізації намічених завдань було обрано систему комп'ютерної алгебри \textsf{GAP} \cite{GAP4}.

Розробка системи \textsf{GAP}, назва якої розшифровується як ''Groups, Algorithms and Programming'', була розпочата у 1986 року в місті Аахен, Німеччина. В 1997 році центр координації розробки і технічної підтримки користувачів перемістився в Університет Сент Ендрюс (Шотландія). На даний момент \textsf{GAP} є унікальним спільним науковим проектом, що об'єднує спеціалістів в області алгебри, теорії чисел, математичної логіки, інформатики та інших наук із різних країн світу. Основні центри розробки знаходяться в університетах таких  міст як Сент-Ендрюс (Шотландія), Аахен та Брауншвейг (Німеччина), та Університеті штату Колорадо (США). Поточна версія системи - \textsf{GAP} 4.10.0, була випущена в листопаді 2018 року.

Спершу система \textsf{GAP} розроблялася під Unix, а потім була портована для роботи в інших операційних системах. На сьогодні вона доступна для роботи в різних версіях Unix/Linux, а також в Windows i Mac OS. Відмітимо, що ряд пакетів, що розширюють функціональність системи, працюють тільки в середовищі Unix/Linux.

\textsf{GAP} є вільно поширюваною, відкритою і доступною для розширення системою. Вона поширюється у відповідності до GNU Public Licence. Система поставляється разом із сирцями, які написані двома мовами: ядро системи написано на С, а бібліотека функцій -- на спеціальній мові, яка теж називається \textsf{GAP}, що за синтаксисом нагадує Pascal, але є об'єктно-орієнтованою мовою. Користувачі можуть створювати свої власні програми на цій мові, і тут сирці є незамінним унаочненням. Зрештою, розробники програм для \textsf{GAP} можуть оформити свої розробки у вигляді пакета для системи \textsf{GAP} і подати їх на розгляд в Раду \textsf{GAP}. Після проходження процедури рецензування і схвалення радою \textsf{GAP}, такий пакет включається до дистрибутиву \textsf{GAP} і поширюється разом із ним. Процедура рецензування дозволяє прирівнювати прийняті Радою \textsf{GAP} пакети до наукової публікації, і посилатися на них так само, як і на інші джерела.

Окрім вже згаданих пакетів, система складається із наступних чотирьох основних компонентів:

\begin{itemize}[noitemsep,partopsep=0pt,topsep=0pt,parsep=0pt]
\item ядра системи, що забезпечує підтримку мови \textsf{GAP}, роботи із системою в програмному і інтерактивному режимі;
\item бібліотеки функцій, в якій реалізовані різноманітні алгебраїчні алгоритми (більше $4000$ користувацьких функцій, більше $140000$ рядків програм на мові \textsf{GAP};
\item бібліотеки даних, включаючи, наприклад, бібліотеку всіх груп порядку не більше $2000$ (за винятком $49487365422$ груп порядку $1024$, точну кількість яких, до речі, було визначено за допомогою системи \textsf{GAP} \cite{konovalov2014}), бібліотеку примітивних груп підстановок, таблиці характерів скінченних груп і т.д., що загалом являє собою ефективний інструмент для висунення та тестування наукових гіпотез;
\item обширної (в районі півтори тисяч сторінок) документації, доступної у різноманітних форматах, а також через інтернет.
\end{itemize}

Систему \textsf{GAP} було задумано як інструмент комбінаторної теорії груп -- розділу алгебри, що вивчає групи, задані твірними елементами і визначальними співвідношеннями. В подальшому, з виходом кожної нової версії системи, сфера її застосування охоплювала усе більше розділів алгебри. В різноманітті областей алгебри, що охоплюються \textsf{GAP} на сьогодні, можна переконатися, прочитавши назви розділів обширної документації по системі.

\textsf{GAP} дає можливість виконувати обчислення з гігантськими цілими і раціональними числами, допустимі значення яких обмежені тільки об'ємом доступної пам'яті. Система дозволяє працювати із циклотомічними полями, скінченними полями, $p$-адичними числами, многочленами від багатьох змінних, раціональними функціями, векторами та матрицями. Користувачеві доступні різні комбінаторні функції, елементарні теоретико-числові функції, різноманітні функції для роботи із множинами та списками.

Групи можуть бути задані в різній формі, наприклад як групи підстановок, матричні групи, групи, задані твірними елементами і визначальними співвідношеннями. Більш того, побудувавши, наприклад, групову алгебру, можна обчислити її мультиплікативну групу, і навіть задати її підгрупу, породжену конкретними оборотними елементами групової алгебри. Ряд груп може бути заданий безпосереднім звертанням до бібліотечних функцій (наприклад, симетрична і знакозмінна групи, циклічна група та ін.).

Теорія представлень груп також входить в область застосування \textsf{GAP}. Тут є інструменти для обчислення таблиць характерів конкретних груп, дій над характерами і інтерактивної побудови таблиць характерів, визначення теоретико-групових властивосте на основі властивостей таблиці характерів групи.

Інформація про нові розробки для застосування у тій чи іншій проблемній області може бути знайдена на сайті \textsf{GAP} \cite{GAP4}.

\section{Інструкція користувача}
\label{section.3.2}

Для реалізації рандомізованих методів для комп'ютерної алгебри, було створено бібліотеку \textsf{random\_methods.g} для системи \textsf{GAP}.

Для підключення бібліотеки, необхідно вказати шлях до неї за допомогою команди  \textsf{ChangeDirectoryCurrent("C:/path/to");}. При цьому слід врахувати, що незалеєно від ОС, \textsf{GAP} приймає на вхід тільки тільки "/" в шляху. Після цього слід викликати команду  \textsf{Read("random\_methods.g");}

Розглянемо деякі функції, реалізовані у модулі.

Лістинг \ref{iscomp} містить ілюстрацію роботи рандомізованого алгоритму верифікації, чи є задане натуральне число складеним \cite{SolStr77}. На вхід подається число $n \in \mathbb{N}$, виходом є true якщо число є складеним, false -- якщо число найімовірніше є простим. Алгоритм полягає у випадковому виборі деякого числа із проміжку  $[1, n-1]$ і перевіряється чи ділиться $n$ на дане число без остачі. Якщо так -- то число є складеним, роботу алгоритма завершено. В іншому ж випадку ми беремо інше випадкове число, поки не буде знайдено дільник $n$, або не буде перевищено число ітерацій.

\begin{lstlisting}[label=iscomp,caption=Функція IsComposite]
gap> IsComposite(524287);
false
gap> time;
7
gap> IsPrime(524287);    
true
gap> IsComposite(524288);
true
gap> time;               
3
gap> IsPrime(524288);    
false
gap> time;           
3
\end{lstlisting}

Лістинг \ref{sqroot} містить ілюстрацію роботи рандомізованого алгоритму знаходження квадратних коренів заданого числа над полем Галуа.
\begin{lstlisting}[label=sqroot,caption=Функція SquareRoot]
gap> F:=GF(19);; el:=Elements(F);;
gap> List(el,i->Int(i));
[ 0, 1, 2, 4, 8, 16, 13, 7, 14, 9, 18, 17, 15, 11, 3, 6, 12, 5, 10 ]
gap> a:=el[6];;                               
gap> sq:=SquareRoot(a,F);; List(sq,i->Int(i));
[ 15, 4 ]
gap> 15^2 mod 19;
16
\end{lstlisting}

Деталі даного алгоритму описано у розділі \ref{ch2}.

Лістинг \ref{cgen} містить ілюстрацію роботи рандомізованого алгоритму знаходження генератора довільної циклічної групи.
\begin{lstlisting}[label=cgen,caption=Функція GeneratorOfCyclicGroupRandom]
gap> C:=CyclicGroup(IsFpGroup,1000);
<fp group of size 1000 on the generators [ a ]>
gap> g:=GeneratorOfCyclicGroupRandom(C);
a^143
\end{lstlisting}

На вхід подається циклічна група $C$ та максимальна кількість ітерацій $n$. На виході отримуємо генератор даної групи. Алгоритм працює наступним чином. Перш за все ми факторизуємо порядок групи (кількість її елементів), тобто подаємо її у вигляді добутку простих чисел. Далі випадковим чином вибираємо елемент $g \in C$ і перевіряємо для кожного простого множника $k$ рівність $g^{\frac{m}{k}}=e$, де  $m =  |C|$ -- порядок групи. Якщо дана рівність виконується, зупиняємо алгоритм, виводимо $g$. В протилежному випадку продовжуємо роботу алгоритму, поки не буде знайдено генератор, або поки не буде перевищено максимальну кількість ітерацій.

Наступний лістинг містить ілюстрацію роботи деяких рандомізованих алгоритмів для групових алгебр.

\begin{lstlisting}[label=units,caption=Рандомізовані алгоритми для групових алгебр]
gap> G:=CyclicGroup(4);
<pc group of size 4 with 2 generators>
gap> KG:=GroupRing(GF(7),G);
<algebra-with-one over GF(7), with 2 generators>
gap> u:=RandomUnit(KG);;  # - повертає випадкову одиницю (оборотний елемент) KG 
gap> Augmentation(u);
Z(7)^4
gap> u*u^-1;
(Z(7)^0)*<identity> of ...

gap> G:=DihedralGroup(IsFpGroup,16);;
gap> KG:=GroupRing(GF(2),G);;
gap> u:=RandomNormalizedUnit(KG);; # повертає випадкову одиницю KG, сума коефіціентів якої рівна 1
gap> Augmentation(u);
Z(2)^0
gap> u*u^-1;
(Z(2)^0)*<identity ...>

gap> G:=CyclicGroup(4);;
gap> KG:=GroupRing(GF(2),G);;
gap> u:=RandomUnitaryUnit(KG);; # повертає випадкову одиницю KG, для якої u*Involution(u) = 1
gap> u*Involution(u);
(Z(2)^0)*<identity> of ...

gap> G:=CyclicGroup(4);;
gap> KG:=GroupRing(GF(2),G);;
gap> u:=RandomNormalizedUnitaryUnit(KG);; # повертає випадкову одиницю KG, для якої u*Involution(u) = 1
gap> u*Involution(u);
(Z(2)^0)*<identity> of ...
gap> Augmentation(u);
Z(2)^0


gap> G:=CyclicGroup(IsFpGroup,4);;
gap> KG:=GroupRing(GF(2),G);;
gap> u:=RandomCentralNormalizedUnit(KG);; # повертає нормалізовану одиницю KG, яка комутує з усіма іншими елементами
gap> Augmentation(u);
Z(2)^0
gap> bool:=true;
true
gap> for x in Elements(KG) do
> if x*u<>u*x then bool:=false; break; fi;
> od;
gap> bool;
true
\end{lstlisting}

Наведемо опис даних функцій:
\begin{itemize}[noitemsep,partopsep=0pt,topsep=0pt,parsep=0pt]
\item \textsf{RandomUnit(KG)} -- для заданого групового кільця (алгебри) $KG$ повертає випадкову одиницю $u \in KG$, тобто такий елемент $u$, для якого існує обернений -- $u^{-1}$;
\item \textsf{RandomNormalizedUnit(KG)} -- для заданого групового кільця (алгебри) $KG$ повертає випадкову одиницю $u \in KG$, сума коефіціентів якої рівна 1;
\item \textsf{RandomUnitaryUnit(KG)} -- для заданого групового кільця (алгебри) $KG$ повертає випадкову одиницю $u \in KG$, для якої її інволюція співпадає з $u^{-1}$;
\item \textsf{RandomNormalizedUnitaryUnit(KG)} -- для заданого групового кільця (алгебри) $KG$ повертає випадкову нормалізовану одиницю $u \in KG$, для якої її інволюція співпадає з $u^{-1}$;
\item \textsf{RandomCentralNormalizedUnit(KG)} -- для заданого групового кільця (алгебри) $KG$ повертає випадкову нормалізовану одиницю $u \in KG$, яка комутує з усіма іншими елементами

\end{itemize}

Дані алгоритми побудовані за тим самим принципом, що і алгоритм знаходження квадратних коренів. Випадковим чином вибирається елемент групової алгебри, для якого перевіряється, чи виконується задана властивість, чи ні. Слід відмітити, що прямий пошук таких елементів може бути дуже витратним за часом, адже навіть для порівняно невеликих груп та полів, групові алгебри можуть містити велику кількість елементів. Наприклад, нехай $G=C_8$ -- циклічна група восьмого порядку, $K=GF(3)$ -- поле Галуа із трьох елементів. Маємо, що
\[
|G| = 8, \quad |K| = 3, \quad |KG| = 3^8 = 6561.
\]
Тож чим більші розмірності групи та поля, тим більш практичним є застосування рандомізованих методів.

Для некомутативної скінченної 2-групи $G$ і натурального числа $n \in \mathbb{N}$, яке позначає максимальну кількість ітерацій, функції \textsf{RandomDihedralDepth} та  \textsf{RandomQuaternionDepth} повертають діедральну та кватерніонну глибину $G$, тобто найбільше таке $d \in \mathbb{N}$, таке, що $G$ містить підгрупу, ізоморфну до діедральної або узагальненої кватерніонної групи розмірності $2^{d+1}$, відповідно.

\begin{lstlisting}[label=units,caption=Рандомізована діедральна та кватерніонна глибина]
gap> G:=SmallGroup([64,6]);
<pc group of size 64 with 6 generators>
gap> StructureDescription(G);
"(C8 x C4) : C2"
gap> RandomDihedralDepth(G,1000);
2


gap> G:=SmallGroup([64,10]);
<pc group of size 64 with 6 generators>
gap> StructureDescription(G);
"(C8 : C4) : C2"
gap> RandomQuaternionDepth(G,1000);
2
\end{lstlisting}

В основі даних методів лежить випадковий вибір двох некомутуючих елементів, і знаходження розмірності групи, яку вони генерують.