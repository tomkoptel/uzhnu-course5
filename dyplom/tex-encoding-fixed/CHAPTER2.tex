\newpage
\renewcommand{\proofname}{Доведення}
\renewcommand{\chaptername}{РОЗДІЛ}
\chapter{Рандомізовані алгоритми комп'ютерної алгебри}
\label{ch2}

\section{Основні поняття рандомізованих алгоритмів}
\label{section.2.1}
\textbf{Рандомізованим алгоритмом} називається алгоритм, який крім вхідних даних, отримує на вхід також потік випадкових бітів, які можуть бути використані для прийняття випадкових рішень \cite{Karp1991}. Навіть для фіксованих вхідних даних, рандомізований алгоритм може видати різний результат. Тож для опису рандомізованих алгоритмів часто використовується апарат теорії ймовірностей. Наприклад, навіть при фіксованих вхідних даних, результат рандомізованого алгоритму є випадковою величиною.

Наведемо деякі базові підходи рандомізованих алгоритмів.

\textit{Достатня кількість свідків}. Рандомізовані алгоритми часто полягають у вирішенні, чи задовільнють вхідні дані певним властивостям (наприклад, чи є факторизувати задане число). Часто, наявність тієї чи іншої властивості можна верифікувати шляхом знаходження певного об'єкту, який називають \textbf{свідком}. Хоча може бутии важко знайти свідка детерміністично, та часто є можливим показати, що свідків є достатньо у певному ймовірнісному просторі, а отже користувач може ефективно шукати свідка шляхом ітеративного вибору елементів із ймовірнісного простору. Якщо властивість виконується, то свідка з високою ймовірністю буде знайдено за кілька ітерацій. Отже, нездатність алгоритму знайти свідка за достатньо велику кількість ітерацій може слугувати доказом (але не строгим доведенням), що вхідні дані не задовольняють бажаній властивості.

\textit{Заплутування супротивника}. Для розуміння переваг рандомізованих алгоритмів може бути корисним теоретико-ігровий підхід. Обчислювальну складність алгоритму можна уявити у вигляді гри із нульовою сумою для двох осіб, у якій один гравець обирає алгоритм, а інший, якого називатимемо \textbf{супротивником}, вибирає вхідні дані, щоби заплутати алгоритм. Виграш супротивника рівний часу виконання алгоритму на даних, наданих супротивником. Рандомізований алгоритм може розглядатися як ймовірнісний розподіл над детерміністичними алгоритмами, а отже як змішана стратегія для гравця, що обирає алгоритм. Вибір змішаної стратегії створює невизначеність, а отже і заплутує супротивника, заважаючи йому вибрати вхідні дані, які створять незручність для алгоритму.

\textit{Відбитки пальців}. Ця техніка дозволяє представляти великі об'єкти за допомогою коротких \textbf{відбитків пальців}. При певних умовах, факт, що два об'єкти мають однаковий відбиток пальців означає, що вони є ідентичними.

\textit{Перевірка тотожностей}. Часто можливо перевірити, чи рівний деякий алгебраїчний вираз нулеві , за допомогою підставляння у нього випадкових чисел, і перевірки, чи отриманий вираз рівниий нулеві. Якщо хоч раз виникне ненульове значення, це означатиме, що вираз не є тотожністю. Якщо ж весь час ми будемо отримувати нуль, то це буде серйозним доказом щодо того, що тотожність виконується.

\textit{Випадкові вибір елементів, впроядкування та розбиття}. Для деяких алгоритмів є корисним рандомізувати порядок, у якому розглядаються вхідні дані. 

\textit{Балансування навантаження}. У контексті розпаралелювання обчислень, рандомізація може застосовуватися для вирівнювання обчислювального навантаження для різних обчислювальних одиниць.

Як можна бачити, вищенаведені властивості дозволяють широке використання рандомізованих підходів.

\section{Ймовірнісні алгоритми комп'ютерної алгебри}
\label{section.2.2}

Однією із перших робіт, присвячених рандомізованим методам в алгебрі була стаття угорських математиків Ердеша і Реньї \cite{erdos65}. Також рандомізовані методи було розглянуто у пакеті \cite{laver20}.

Основними задачами ймовірнісної теорії груп є:

\begin{itemize}[noitemsep,partopsep=0pt,topsep=0pt,parsep=0pt]
\item Альтернативний опис структури груп та їх елементів за допомогою апарату теорії ймовірностей;
\item Застосування ймовірнісних методів для доведення детерміністичних теорем.
\end{itemize} 

За допомогою ймовірнісних методів можна розв'язати чимало цікавих задач. Розглянемо деякі із них.

Нехай $G$ -- скінченна, некомутативна група. Якою є ймовірність того, що два випадковим чином вибрані елементи комутують?

Припускаючи, що два елементи вибираються незалежно і кожна пара елементів має ту саму ймовірність $\frac{1}{|G|^2}$ бути обраною, вищенаведене питання може бути інтерпретоване, як питання обчислення
\[
p = \frac{|\{(x,y) \in G\times G: xy =yx\}|}{|G|^2}.
\]

Шляхом рандомізованих обчислень, можна виявити, що $p\leq \tfrac{5}{8}$.

Нехай $S_n$ -- симетрична група порядку $n$ (множина усіх перестановок з $n$ елементів). Відомо, що симетрична група генерується двома елементами. Питання: скільки пар елементів $S_n$ генерують усю групу?

Діксоном було доведено, що ймовірність $p_n$, що два випадковим чином вибрані елементи $S_n$ генерують $S_n$ прямує до одиниці, при $n \to \infty$. Більше того, Діксон також показав, що якщо $G$ -- скінченна проста група, то ймовірність того, що два випадковим чином вибрані елементи цієї групи генерують її, прямує до 1 при $|G|\to \infty$.

Також рандомізовані алгоритми можуть бути ефективно використані для знаходження коренів у скінченних полях.

Нехай $p$ -- просте число, і нехай $\mathbb{Z}_p^{*} = \{1,2,\ldots,p-1$ -- мультиплікативна група поля $GF(p)$. Елемент $a  \in \mathbb{Z}_p^{*}$ називається \textbf{квадратним лишком}, якщо він є квадратом якогось елемента із  $\mathbb{Z}_p^{*}$, тобто, якщо існує $z\in \mathbb{Z}_p^{*}$ такий, що $z^2 = a$. \textbf{Символ Лежандра} $(a/p)$ для елемента $a  \in \mathbb{Z}_p^{*}$ є індикатором того, чи є $a$ квадратним лишком (у цьому випадку значення рівне 1), чи ні (значення -- -1). Символ Лежандра має мультиплікативну властивість:
\[
(a_1a_2/p) = (a_1/p)(a_2/p).
\] 
Також, із того факту, що $\mathbb{Z}_p^{*}$ є циклічною групою, випливає те, що $(a/p)=a^{(p-1)/2}$. Це показує, що символ Лежандра легко обчислити.

Припустимо, що ми знаємо, що $a$ є квадратним лишком у $\mathbb{Z}_p^{*}$ і ми хочемо знайти квадратні корені $a$. Іншими словами, ми хочемо факторизувати поліном $x^2-a$ над $\mathbb{Z}_p^{*}$. Втім, достатнім є отримати не стільки факторизацію полінома $x^2 - a$, а факторизацію полінома $(x-c)^2-a$ для деякого $c \in \mathbb{Z}_p^{*}$, адже це зводиться тільки до зсуву коренів $x^2 - a$. Тож припустимо, що
\[
(x-c)^2 - a = (x-r)(x-s).
\]
Тоді $rs=c^2-a$ і $(r/p)(s/p)=((c^2-a)/p)$. Якщо, після вибору $c$ і обчислення $((c^2-a)/p)$, виявиться, що $((c^2-a)/p)\neq 1$, то із мультиплікативної властивості символу Лежандра випливає те, що або $r$, або $s$ є квадратним лишком. З іншого боку, квадратичні лишки у $\mathbb{Z}_p^*$ є коренями полінома $x^{(p-1)/2}-1$, а отже найбільшим спільним дільником $(x-c)^2-a$ і $x^{(p-1)/2}-1$ є поліном першого степеня, тож ми випадковим чином вибираємо $c$, перевіряємо чи $c^2-a$ є квадратним лишком і, якщо не є, за допомогою нескладних обчислень отримуємо корінь $(x-a)^2-a$, з якого ми отримуємо $\sqrt{a}$. Тож алгоритм виглядає наступним чином:

\begin{itemize}[noitemsep,partopsep=0pt,topsep=0pt,parsep=0pt]
\item Вибираємо випадкове $c \in \mathbb{Z}_p^*$;
\item Якщо $((c^2-a)/p)=-1$, обчислюємо $gcd(x^{(p-1)/2}-1,(x-c)^2-a)$. Результатом буде $\alpha x - \beta$ і нулем $(x-c)^2-a$ буде $r=\alpha^{-1} \beta$. Повертаємо $\sqrt{a} = \pm (c+r)$.
\end{itemize}   

Основним питанням є, наскільки достатньо є таких елементів $c$ таких, що $c^2-a$ є квадратним лишком? Виявляється \cite{Karp1991}, що ймовірність вибрати такий елемент із мультиплікативної групи поля є більшою за $\tfrac{1}{2}$. 

Даний алгоритм є прикладом рандомізованого алгоритму, який спирається на достатність свідків. Це є алгоритм Лас-Вегас, тобто він дає розв'язок із ймовірністю, більшою за $\tfrac{1}{2}$ і ніколи не хибить. Втім, часто нам доводиться задовольнятися слабшим результатом: алгоритмом Монте-Карло, який, якщо відповідь існує, підтверджує її з ймовірністю, більшою за $\tfrac{1}{2}$, а якщо відповіді нема -- не повертає нічого.