\Conclusion % заключение к отчёту

Моделі можуть мати наступні недоліки:

\begin{itemize}
    \item \textbf{переобладнання}: захоплення помилкових патернів, які не повторяться в майбутньому, що призведе до менш точних прогнозів, або
    \item \textbf{недооснащення}: не вдалося вловити відповідні закономірності, що знову призводить до менш точних прогнозів.
\end{itemize}
Ми використовуємо дані перевірки, які не використовуються при навчанні моделей, для вимірювання точності кандидатської моделі.
Це дозволяє нам спробувати багато моделей кандидатів і, в кінці кінців, вибрати найкращу.

Є параметри, які дозволяють поліпшити нам ефективність випадкового лісу настільки, наскільки ми змінимо значення максимальної глибини одного дерева рішень.
Але однією з найкращих особливостей моделей Random Forest є те, що вони, як правило, прогнозують значення дуже добро навіть без додаткового налагоджування.
