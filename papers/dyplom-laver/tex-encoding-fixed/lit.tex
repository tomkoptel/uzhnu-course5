\newpage
\addcontentsline{toc}{chapter}{\bf Список використаної літератури}
\renewcommand{\bibname}{\centerline{\bf Список використаної літератури}}
%\chapter*{\centerline{\bf СПИСОК ВИКОРИСТАНОЇ ЛІТЕРАТУРИ}}\label{lit}

\renewcommand{\refname}{\centerline{\bf Список використаної літератури}}
\begin{thebibliography}{999}\label{bib}
%{ \baselineskip=20pt
%1


\bibitem{Karp1991}  
{\it Richard M. Karp} An introduction to randomized algorithms / R.M. Karp // 
Discrete Applied Mathematics, Volume 34, Issues 1–3, 1991,pp. 165--201.

\bibitem{SolStr77} {\it R. Solovay and V. Strassen} A fast Monte-Carlo test for primality / R. Solovay, V. Strassen // SIAM J. Comput., 6, 1977, pp. 84--85.


\bibitem{Rabin37} {\it M.O. Rabin} Probabilistic algorithms / M.O. Rabin // in: J. Traub, ed., Algorithms and Complexity. - Academic Press, New York, 1976.


\bibitem{Gill77} {\it J. Gill} Computational complexity of probabilistic Turing machines/ J. Gill // SIAM J. Comput., 6, 1977, pp. 675--695.

\bibitem{Solomon2001} {\it Solomon, Ronald} A brief history of the classification of the finite simple groups / Solomon R. // American Mathematical Society. Bulletin. New Series, 38 (3), 2001, pp. 315--352. 

\bibitem{Griess82}  
{\it Griess, Robert L.} The friendly giant // Griess, R.L./ Inventiones Mathematicae, 69 (1), 1982, pp. 1--102. 

\bibitem{Wilson99}
{\it Wilson, Robert A.} The maximal subgroups of the Baby Monster. I // Wilson R.A. / Journal of Algebra, 211 (1), 1999, pp. 1--14,

\bibitem{Lang2002}
{\it Lang S.} Algebra / Serge Lang. – New York: Springer - Verlag, 2002. – 914 с. 

\bibitem{erdos65}
{\it Erdos, P., Renyi} A. Probabilistic methods in group theory. // Erdos, P., Renyi A. / J. Anal. Math. 14, 1965, 127--138. 

\bibitem{laver20}
{\it Balogh, Z. A. and Laver, V.} RAMEGA, A Package for Random Methods in Group Algebras. Version 1.000, 2020. URL: https://vlaver.github.io/RAMEGA. 

\bibitem{konovalov2014}
{\it Коновалов А.} Cистема компьютерной алгебры GAP 4.7 [Електронний ресурс] / Александр Коновалов. – 2014. – Режим доступу до ресурсу: https://www.gap-system.org/ukrgap/gapbook/manual.pdf.
%}


%{ \baselineskip=20pt

  \bibitem{GAP4}
  The GAP~Group, \emph{GAP -- Groups, Algorithms, and Programming, 
  Version 4.10.0}; 
  2018,
  \verb+(https://www.gap-system.org)+.
%}

\end{thebibliography}



\newpage
\addcontentsline{toc}{chapter}{\bf Додаток. Текст програми}

\chapter*{\centerline{\bf Додаток. Текст програми}}\label{vstyppp}

\lstset{ %
language=GAP,                 % выбор языка для подсветки (здесь это С)
basicstyle=\small\sffamily\linespread{0.8}, % размер и начертание шрифта для подсветки кода
%numbers=left,               % где поставить нумерацию строк (слева\справа)
%numberstyle=\tiny,           % размер шрифта для номеров строк
%stepnumber=1,                   % размер шага между двумя номерами строк
%numbersep=5pt,                % как далеко отстоят номера строк от подсвечиваемого кода
backgroundcolor=\color{white}, % цвет фона подсветки - используем \usepackage{color}
showspaces=false,            % показывать или нет пробелы специальными отступами
showstringspaces=false,      % показывать или нет пробелы в строках
showtabs=false,             % показывать или нет табуляцию в строках
%frame=single,              % рисовать рамку вокруг кода
tabsize=2,                 % размер табуляции по умолчанию равен 2 пробелам
captionpos=t,              % позиция заголовка вверху [t] или внизу [b] 
breaklines=true,           % автоматически переносить строки (да\нет)
breakatwhitespace=false, % переносить строки только если есть пробел
escapeinside={\%*}{*)}   % если нужно добавить комментарии в коде
}

\fontsize{12}{14}\selectfont

\begin{lstlisting}[label=dod, frame=none]
SquareRoot:=function(a,F)
  local c,x,alpha,beta,r,p,G, pol,coef;
  p:=Size(F);
  G:=Group(PrimitiveElement(F));

  c:=Random(Elements(G));
  while Legendre(Int(c)^2-Int(a),p)<>-1 do
    c:=Random(Elements(G));
  od;

  x:=Indeterminate(F);
  pol:=Gcd(x^((p-1)/2)-1,(x-c)^2 -a);
  coef:=CoefficientsOfUnivariatePolynomial(pol);
  alpha:=coef[2];
  beta:=coef[1];
  r:=alpha^(-1)*beta;

  return [c+r,-c-r];
end;

IsComposite:=function(n)
  local a,i;
  i:=1;
  while i<=1000 do
    a:=Random([1..n-1]);
    if Gcd(a,n)<>1 then
      return true;
    fi;
    i:=i+1;
  od;
  return false;
end;


#Computes the generator of the CyclicGroup in a Random way
GeneratorOfCyclicGroupRandom:=function(G)
  local fact,i,bool,n,m,g,bool2;
  if IsCyclic(G)=false then return fail; fi;
  fact:=Set(FactorsInt(Size(G)));

  n:=Size(fact);m:=Size(G);
  bool:=false;

  while bool=false do
    g:=Random(G);
    bool2:=false;
    for i in [1..n] do
      if g^(m/fact[i])=One(G) then bool2:=true; break; fi;
    od;
    if bool2 = false then bool:=true; fi;
  od;
  return g;
end;


RandomUnit:=function(kg)
   local x,o;
   o:=Zero(UnderlyingField(kg));
   repeat
      x:=Random(kg);
   until(x^-1<>fail);
   return x;
end;

RandomNormalizedUnit:=function(kg)
local k, g, e, x;
  k:=UnderlyingField(kg);
	g:=UnderlyingGroup(kg);
  if not(Size(g) mod Characteristic(k) = 0) then
	   Error("Input should be a Modular Group Algebra.");
 fi;

 e:=One(UnderlyingField(kg));
 repeat
    x:=Random(kg);
 until(Augmentation(x) = e);
 return x;
end;

RandomUnitaryUnit:=function(kg)
   local x,e;
   e:=One(kg);
   repeat
      x:=RandomUnit(kg);
   until(x*Involution(x) = e);
   return x;
end;

RandomNormalizedUnitaryUnit:=function(kg)
   local x,e,k,g;
   k:=UnderlyingField(kg);
 	g:=UnderlyingGroup(kg);
   if not(Size(g) mod Characteristic(k) = 0) then
 	   Error("Input should be a Modular Group Algebra.");
  fi;
   e:=One(kg);
   repeat
      x:=RandomNormalizedUnit(kg);
   until(x*Involution(x) = e);
   return x;
end;

RandomCentralNormalizedUnit:=function(kg)
   local x,e,c,k,g;
   k:=UnderlyingField(kg);
 	g:=UnderlyingGroup(kg);
   if not(Size(g) mod Characteristic(k) = 0) then
 	   Error("Input should be a Modular Group Algebra.");
  fi;
   c:=Center(kg);
   e:=One(UnderlyingField(kg));
   repeat
      x:=Random(c);
   until(Augmentation(x) = e);
   return x;
end;

IsDihedralGroup := function(G)
local p,m,exponent;

    if ( IsPGroup(G) and not(IsAbelian(G)) ) then
      p:=PrimePGroup(G);
      if p<>2 then
        # Error("The Group should be a non abelian 2-group.");
        return false;
      fi;
      exponent:=Lcm(Set((List(Elements(G),j->Order(j)))));
      if exponent=Size(G)/2 then
        if Size(Filtered(Elements(G),j->Order(j)<=2))=exponent+2 then
          return true;
        else
          return false;
        fi;
      else return false;
      fi;
    else
      return false;
    fi;
end;


RandomDihedralDepth:=function(KG,n)
    local a,b,dd,g,s,k,i,j,exp;
    if not( IsPGroup(UnderlyingGroup(KG))) then
        Error("G should be a p group.");
    fi;

    k:=0;
    for i in [1..n] do
      a:=RandomUnit(KG); b:=RandomUnit(KG);
      if Order(a)>=2 and Order(b)>=2 and a*b<>b*a then
        g:=Group(a,b); s:=Size(g);
       if IsDihedralGroup(g) and s>k then
           k:=s;
       fi;
      fi;
    od;

    if k<>0 then return LogInt(k,2)-1; else return 0; fi;
end;

RandomQuaternionDepth:=function(KG,n)
    local a,b,dd,g,s,k,i,j,exp;
    if not( IsPGroup(UnderlyingGroup(KG))) then
        Error("G should be a p group.");
    fi;
    k:=0;
    for i in [1..n] do
      a:=RandomUnit(KG); b:=RandomUnit(KG);
      if Order(a)>=2 and Order(b)>=2 and a*b<>b*a then
        g:=Group(a,b); s:=Size(g);
        exp:= Lcm(Set(List(Elements(g),j->Order(j))));

        if s/exp=2 then
          if Size(Filtered(Elements(g),j->Order(j)<=2))=2 and s>k then k:=s; fi;
        fi;
      fi;
    od;

    if k<>0 then return LogInt(k,2)-1; else return 0; fi;
end);

\end{lstlisting}

%%================
