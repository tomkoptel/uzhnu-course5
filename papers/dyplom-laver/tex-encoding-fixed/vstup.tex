\newpage
\pagenumbering{arabic}
\setcounter{page}{3}
\addcontentsline{toc}{chapter}{\bf Вступ}

\chapter*{\centerline{\bf Вступ}}\label{vstyp}

Чимало сучасних задач практики можуть бути ефективно вирішені за допомогою рандомізоаних (ймовірнісних, випадкових) алгоритмів \cite{Karp1991}. Одними із перших робіт, присвячених рандомізованим алгоритмам, були робота Соловая та Штрассена \cite{SolStr77} та стаття Рейбіна \cite{Rabin37}, які привернули увагу до загальної концепції рандомізованих алгоритмів і дали кілька гарних застосувань для теорії чисел та обчислювальної геометрії. Також вартою уваги є робота Джілла \cite{Gill77}, яка заклала основи для розширення абстрактної теорії складності на випадок рандомізованих алгоритмів.

На сьогодні визнається те, що у широкому спектрі застосувань, рандомізація є надзвичайно важливим інструментом для розробки алгоритмів. Є два основних типи переваг, які дають рандомізовані алгоритми:

\begin{itemize}[topsep=0pt, partopsep=0pt, itemsep=0pt]
 
\item По-перше, часто час виконання рандомізованого алгоритму та вимоги щодо пам'яті є меншими, ніж навіть у кращих детерміністичних алгоритмів для тієї самої задачі. 

\item По-друге, якщо подивитися на більшість розроблених рандомізованих алгоритмів, можна побачити, що вони є надзвичайно простими для розуміння. Часто введення рандомізації зводиться до перетворення простого і наївного детерміністичного алгоритму із поганою поведінкою у найгіршому випадку, до рандомізованого алгоритму, що показує хорошу продуктивність із високою ймовірністю для будь-яких вхідних даних.
\end{itemize}

Також, останні десятиліття відзначилися стрімким розвитком алгебри. Було доведено теорему про класифікацію скінченних простих груп \cite{Solomon2001} (це доведення займає близько 15 тис. сторінок), відкрито такі групи як Монстр \cite{Griess82}, та Маленький Монстр (Baby Monster) \cite{{Wilson99}}, кількість елементів у яких приблизно рівна $8 \cdot 10^{53}$ та $4 \cdot 10^{33}$, відповідно. Враховуючи складність обчислень та практичну незастосовність детерміністичних методів через величезний об'єм даних, \textbf{актуальним} є використання рандомізованих методів комп'ютерної алгебри. 

\textbf{Метою} даної магістерської роботи є дослідження ймовірнісних методів комп'ютерної алгебри. 

Для досягнення цієї мети, було поставлено наступні \textbf{завдання}:
\begin{itemize}[noitemsep,partopsep=0pt,topsep=0pt,parsep=0pt]
\item Зробити огляд рандомізованих методів комп'ютерної алгебри;
\item Розробити програмний продукт для реалізації даних методів;
\item Перевірити програмний продукт на тестових прикладах.
\end{itemize}

Дана магістерська робота складається із трьох розділів і додатку. У першому розділі наведено основні означення із алгебри. Другий розділ містить опис ранжомізованих методів комп'ютерної алгебри. Третій розділ містить опис практичної частини роботи -- обґрунтування вибору середовища програмування, інструкцію для користувача та результати роботи розробленого програмного продукту на тестових прикладах. В додатку наведено текст розробленої програми.

