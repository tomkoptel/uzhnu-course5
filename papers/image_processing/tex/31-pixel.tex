\chapter{Репрезентація пікселя}\label{cha:ch_1}

\section{Репрезентація пікселя}\label{sec:pixel_definition}
Зображення можно описати концепцією 3-D обєму (I), який можна описати математично:
\[ \alpha = I \rightarrow R, I \subset R \]\cite{img_operators:4}
Таким чином, кожний піксель зображення має дійсне число, яке його репрезинтує.
Однак, в реальності зручніше зберігати та ефективніше опрацьовувати цілі числа замість чисел з плаваючою точкою.
Тобто, значення пікселів описані цілими числами.

\section{Глибина біту (Bit Depth)}\label{sec:bit_depth}
Діапазон значень, який використовується для опису довільного формату зображення визначається глибиною біта.
Діапазон можна описати, як відрізок \([0, 2^{bitdepth - 1}]\).\cite{img_operators:4}
Наприклад, 8-бітне зображення буде мати діапазон \([0, 2^{8} - 1] = [0, 255]\).\cite{img_operators:4}
Таким чином, чим вище значення глибину біту тим більше потрібно дискового простору та пам'яті, щоби опрацювати зображення.
Більшість поширених фото-форматів, таких як jpeg, png тощо, використовують 8-біт для зберігання і набувають лише позитивні значення.

Якщо потрібна більш висока точність (наприклад, наукове застосування), тоді використовують глибину біту вищих значень.
Наприклад, 16-bit зображення буде мати значення пікселів в діапазоні [0, 65535], де  \(65535 = 2^{16}\).\cite{img_operators:4}
Для зображень, що мають як позитивні, так і негативні значення пікселів, діапазон становить [−32768, + 32768].\cite{img_operators:4}
Загальна кількість значень у цьому випадку становить \(65536(=2^{16})\) або бітова глибина 16.\cite{img_operators:4}
Прикладом такого зображення є зображення CT DICOM.

Коли потрібно зробити зображення з вискоким рівнем точності тоді використовують більшу глубину біта.
Наприклад, значення пікселя від 1000 описує колір кістки.
При використанні 8-бітного діапазону використовується значення 255, тобто білий.
Всі значення більші за 255 будуть інтерпольовані, як білий колір, в свою чергу ми втрачаємо точність.

\section{Формат та глибина. Відношення}\label{sec:formats}
\begin{itemize}
    \item DICOM - 1 канали по 16-bit, отже глибина 16-біт.
    \item RGB - 3 канали по 8-bit, отже глибина 24-біта.
    \item TIFF - 5 каналів по 16-bit, отже глибина 80-біта.
\end{itemize}
