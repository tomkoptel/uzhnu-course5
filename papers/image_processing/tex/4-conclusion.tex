\Conclusion % заключение к отчёту

Дана робота розглянула кілька концепцій які використовуються в теорії представлення зображення.
Ми почали наш розгляд з понять глибини біта, пікселя та вокселя.
Діапазон значень, який використовується для опису довільного формату зображення визначається \textbf{глибиною біта}.
Значення пікселів описиують цілими числами, оскільки в реальності зручніше зберігати та ефективніше опрацьовувати цілі числа замість чисел з плаваючою точкою.
Якщо пікслель набуває значення глибини то він називається VOlume piXEL та має розміри XYZ.

Далі ми розглянули систему координат, що використовується в описі зображень векторного та растрового форматів.
Координатна система має початок у верхньому лівому куті, вісь x простягається праворуч, а вісь y - вниз.
Векорний формат описує вміст зображення за допомогою положення та розміру геометричних форм і фігур, таких як лінії, криві, прямокутники та кола
Растрові, зображення є «цифровими фотографіями», вони є найпоширенішою формою для представлення природних зображень та інших видів графіки, які багаті на деталі.
око сприймає форму як різницю переходу між більшим та меньшим значення ваги пікселя (різниця між відтінками).

\textbf{Роздільна здатність} - це вимірювання щільності дискретизації, роздільна здатність растрових зображень дає зв'язок між розмірами пікселів та фізичними розмірами.
Найбільш часто використовується вимірювання - ppi(pixel per inch), пікселів на дюйм.
Також ми дізналися, що найпоширенішим способом моделювання кольорів в комп'ютерній графіці є кольорова модель RGB.
