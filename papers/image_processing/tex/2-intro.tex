\Introduction

Зображення - найкращий спосіб опису дійсності.
Людина на багато краще розпізнає зображення ніж тескст або аудіозапис.
В цьому немає нічого дивного, адже зір та розпізнавання образів є критичним з точки зору виживання.
Томy і не дивно, що основний спосіб взаємодії з комп'ютером, це робота з засобами виводу такми як монітор.

Зображення, визначене з математичної точки зору, вважається функцією двох реальних змінних, наприклад, a(x, y) де "а" - амплітуда (наприклад, яскравість) зображення з координатою (x, у).
Крім того, можна вважати, що зображення містить підзображення, які іноді називають регіонами інтересів, ROI або просто регіонами.
Ця концепція відображає той факт, що зображення часто містять колекції предметів, кожен з яких може бути основою для регіону.

В даній роботі ми розглянемо теорертичний матеріал, щоби відповісти на питання: "Яким чином реалізоване представлення зображення у пам'яті комп'ютера".
Спочатку ми розглянемо загальні відомості.
Далі опишемо поняття піксель, воксель, растрове та векторне зображення.
Обговоримо поняття каналів і опишемо обєми даних використаних для опису зображення.
