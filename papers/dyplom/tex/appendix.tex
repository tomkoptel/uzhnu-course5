\appendix
\renewcommand{\appendixname}{Додаток}


\chapter{Сирцевий код}
\label{cha:appendix1}

\begin{lstlisting}[style=light, language=Python,label={lst:rn_network},caption=React Native Network Call]
const loadFromNetwork = () => {
  return fetch("https://dog.ceo/api/breeds/list/all")
    .then((response) => response.json())
    .then((json) => {
      return items;
    })
    .then((data) => setData(data))
    .catch((error) => console.error(error))
    .finally(() => setLoading(false));
};
\end{lstlisting}

\begin{lstlisting}[style=light, language=Python,label={lst:rn_realm},caption=React Native Realm]
# We define the schema of the table
const BreedSchema = {
  name: "Breed",
  primaryKey: "id",
  properties: {
    id: "string",
    name: "string",
    isFavorite: { type: "bool", default: false },
  },
};

# We do list all schemas available and configure conflict resolution strategy
const config = {
  deleteRealmIfMigrationNeeded: true,
  schema: [BreedSchema],
};

# Finally we do consume the values from the database
const loadFromDatabase = () => {
  return Realm.open(config).then((realm) => {
    const breeds = realm
      .objects("Breed")
      .filtered(`TRUEPREDICATE SORT(name ASC) DISTINCT(name)`);
    # For each item we need detach it from the Realm to avoid accidental mutations
    const items = breeds.map(copyObjects);
    realm.close();
    return items;
  });
};
\end{lstlisting}

\begin{lstlisting}[style=light, language=Python,label={lst:rn_state_hooks},caption=React Native Hooks]
export default function App() {
  # Declaring state of array of items
  const [data, setData] = useState([]);
  # Mark the code as one that defines effect. In this case we read from DB.
  useEffect(() => {
    // 3
    loadFromDatabase().then((data) => setData(data));
  });

  # Return the list of items as widget that transforms data as <Item> widget
  return <FlatList
    data={data}
    renderItem={({ item }) => {
      return <Item item={item} />;
    }}
    keyExtractor={(item) => item.id}
  />
}
\end{lstlisting}


\begin{lstlisting}[style=light, language=Python,label={lst:flutter_app_widget},caption=Flutter StatelessWidget]
# All imported from Dart pubsub packages installed locally
import 'package:flutter/material.dart';
import 'package:provider/provider.dart';
import 'data/breed_database.dart';
import 'home.dart';

Future<void> main() async {
  # Make sure widgets intitialised, so that we can avoid tree state inconsistencies
  WidgetsFlutterBinding.ensureInitialized();

  # We do create an instance that connects to the local SQLite database
  final database = await BreedDatabase.create();

  # runApp kicks off the widget creation
  # we do pass a Provider that delegates database instance
  # to any leaf of widget tree interested in making calls
  runApp(Provider<BreedDatabase>(
    create: (_) => database,
    child: MyApp(),
  ));
}

# It is a root widget instantiated
# marked as one without a state management
class MyApp extends StatelessWidget {
  @override
  Widget build(BuildContext context) {
    # We do use the Material Design widgets
    return MaterialApp(
      # The home widget represented with HomePage
      home: HomePage.create(context, title: 'Flutter List of Breeds'),
    );
  }
}
\end{lstlisting}

\begin{lstlisting}[style=light, language=Python,label={lst:flutter_app_widget},caption=Flutter StatefulWidget]
# We are declaring widget
# and inherit widge that will hold a state object
class BreedList extends StatefulWidget {
  BreedList({Key? key, required this.viewModel}) : super(key: key);
  final BreedListViewModel viewModel;

  @override
  _BreedListState createState() => _BreedListState();
}

# State object holds infor about state
# and uses widge to access dependencies
class _BreedListState extends State<BreedList> {
  @override
  void initState() {
    super.initState();
    # As soon as widget instantiated we do consume list of items
    widget.viewModel.loadBreedList();
  }

  # Lastly we use context to observe chages of the state
  @override
  Widget build(BuildContext context) {
    return context
        .watch<BreedListState>()
        .when(loaded: showList, error: showError, loading: showLoading);
  }
}
\end{lstlisting}

\begin{lstlisting}[style=light, language=Python,label={lst:flutter_networking},caption=Flutter Networking]
import 'dart:convert';

import 'package:breed_flutter/domain/breed.dart';
import 'package:http/http.dart' as http;
import 'package:http/http.dart';

class BreedApi {
  # We are performing call in asynchronous way
  Future<List<Breed>> fetchBreeds() async {
    # Creates the unique resource identifier
    final Uri url = Uri.parse('https://dog.ceo/api/breeds/list/all');
    # Fire the network call and await for response
    final Response response = await http.get(url);
    # Decode response
    final Map<String, dynamic> body = jsonDecode(response.body);
    # Extract value by key
    final Map<String, dynamic> message = body['message'];

    # If 200 OK we proceed with mapping raw response to collection of domain types
    if (response.statusCode == 200) {
      return message.keys.map((breedName) {
        final List<dynamic> rawRaces = message[breedName];
        final List<String> races =
            rawRaces.map((race) => race.toString()).toList();
        return Breed.make(name: breedName, races: races);
      }).toList();
    } else {
      # We perform short-circuit exit as we failed to consume the result
      throw Exception('Failed to load breed items');
    }
  }
}
\end{lstlisting}

\begin{lstlisting}[style=light, language=Python,label={lst:flutter_sqflite},caption=Flutter SQFLITE]
import 'package:breed_flutter/domain/breed.dart';
import 'package:path/path.dart';
# We are using sqflite to avoid implementing low-level DB communication
import 'package:sqflite/sqflite.dart';

# Wrapper class that hides usage of sqflite lib
# Improves encapsulation of code by increasing cohesion
class BreedDatabase {
  final Database database;

  BreedDatabase(this.database);

  # Creates and instance of database returned as a promise
  static Future<BreedDatabase> create() async {
    # Opens connection relying on the default DB path
    final Database database = await openDatabase(
      join(await getDatabasesPath(), 'doggie_database.db'),
      onCreate: (db, version) {
        # RAW SQL statement used to create table
        return db.execute(
          "CREATE TABLE breeds(id TEXT PRIMARY KEY, name TEXT, races TEXT, isFavorite INTEGER)",
        );
      },
      version: 1,
    );
    return BreedDatabase(database);
  }

  # Creates new instance of Breed model
  Future<void> insert(Breed breed) async {
    await database.insert(
      'breeds',
      breed.toMap(),
      conflictAlgorithm: ConflictAlgorithm.replace,
    );
  }

  # Returns list of all breeds
  Future<List<Breed>> all() async {
    final List<Map<String, dynamic>> maps =
        await database.query('breeds', orderBy: 'name');

    var list = List.generate(maps.length, (i) {
      return Breed.fromMap(maps[i]);
    });
    return list;
  }

  # Prunes the contents of the breeds table
  Future<void> clear() async {
    await database.delete('breeds');
  }
}
\end{lstlisting}

\begin{lstlisting}[style=light, language=Python,label={lst:flutter_sqflite},caption=Flutter SQFLITE]
# BreedListViewModel extends StateNotifier<BreedListState>
# where the state described with BreedListState
class BreedListViewModel extends StateNotifier<BreedListState> {
  BreedListViewModel({required this.database})
      : super(const BreedListState.loading());
  final BreedDatabase database;

  Future<void> loadBreedList() async {
      # Extract all items from DB
      final List<Breed> cachedBreeds = await database.all();
      # Assign items to the state
      state = BreedListState.loaded(savedBreeds);
  }
}

class BreedList extends StatefulWidget {
  BreedList({Key? key, required this.viewModel}) : super(key: key);
  final BreedListViewModel viewModel;

  @override
  _BreedListState createState() => _BreedListState();
}

class _BreedListState extends State<BreedList> {
  @override
  void initState() {
    super.initState();
    # As soon as our state widget initialized we do communicate to the view model
    widget.viewModel.loadBreedList();
  }

  @override
  Widget build(BuildContext context) {
    # We do listen to the changes of the view model
    # BreedList depends on BreedListViewModel and uses BreedListState
    # to permutate between all possible states
    return context
        .watch<BreedListState>()
        .when(loaded: showList, error: showError, loading: showLoading);
  }
}
\end{lstlisting}

\begin{lstlisting}[style=light, language=Python,label={lst:flutter_unit_test},caption=Flutter Unit Test]
@GenerateMocks([BreedDatabase, BreedApi])
void main() {
  # First we need to create mocks for DB and network communication
  # This allows us to isolate changes caused by moving parts
  # and keep the logic of the test simple and isolated
  final mockBreedDatabase = MockBreedDatabase();
  final mockBreedApi = MockBreedApi();

  # We need to pass our mocks as dependencies
  final listViewModel =
      BreedListViewModel(database: mockBreedDatabase, api: mockBreedApi);

  # We do create 2 fake objects to represent
  # the state of DB and network data stores
  final fakeDbBreed = Breed.make(name: "fakeDB");
  final fakeNetworkBreed = Breed.make(name: "fakeNetwork");

  # We do use `group` to create a group of related unit tests
  group('when toggle breed item', () {
    # Before each test we do setup expectation for DB always return items
    setUp(() {
      when(mockBreedDatabase.all())
          .thenAnswer((_) => Future.value(<Breed>[fakeDbBreed]));
    });

    test('should toggle breed and load new list items', () async {
      await listViewModel.toggleFavorite(fakeDbBreed);

      # We do capture the value we pass to DB
      var captured =
          verify(mockBreedDatabase.insert(captureAny)).captured.single;
      var capturedBreed = captured as Breed;

      # We do assert that the captured value has been mutated
      # In this case that its flag has been reverted
      expect(capturedBreed.isFavorite, !fakeDbBreed.isFavorite);
    });
  });
\end{lstlisting}

\begin{lstlisting}[style=light, language=Python,label={lst:android_xml},caption=Android UI with XML]
# Activity is an entry point to the app UI
class MainActivity : AppCompatActivity() {
    override fun onCreate(savedInstanceState: Bundle?) {
        super.onCreate(savedInstanceState)
        # We do configure this page to render specific XML layout
        setContentView(R.layout.activity_main)
    }
}

# XML is a declarative way of describing UI on Android
<?xml version="1.0" encoding="utf-8"?>
<androidx.constraintlayout.widget.ConstraintLayout
    xmlns:android="http://schemas.android.com/apk/res/android"
    xmlns:app="http://schemas.android.com/apk/res-auto"
    xmlns:tools="http://schemas.android.com/tools"
    android:layout_width="match_parent"
    android:layout_height="match_parent"
    tools:context=".MainActivity"/>
\end{lstlisting}

\begin{lstlisting}[style=light, language=Python,label={lst:android_jetpack_compose},caption=Android Jetpack Compose]
class MainActivity : AppCompatActivity() {
    override fun onCreate(savedInstanceState: Bundle?) {
        super.onCreate(savedInstanceState)

        val appContainer = (application as JetnewsApplication).container
        # We use Kotlin extension function
        setContent {
            # The enclosing lambda will create our widget and return to the app
            JetnewsApp(appContainer, navigationViewModel)
        }
    }
}
\end{lstlisting}

\begin{lstlisting}[style=light, language=Python,label={lst:ios_swift_ui},caption=Swift UI]
import SwiftUI

# Widget declaration as an entry point to the app
@main
struct LandmarksApp: App {
    # The scene defines our screen
    var body: some Scene {
        WindowGroup {
            # Finally we yield a widget back to the enclosing lambda
            ContentView()
        }
    }
}
\end{lstlisting}

\begin{lstlisting}[style=light, language=Python,label={lst:kmm_ktor},caption=Ktor]
import io.ktor.client.HttpClient
import io.ktor.features.json.*
import io.ktor.request.*
import io.ktor.http.takeFrom

interface KtorApi {
    suspend fun getJsonFromApi(): BreedResult
}

class DogApiImpl : KtorApi {
    private val client = HttpClient {
        install(JsonFeature) { serializer = KotlinxSerializer() }
    }

    init { ensureNeverFrozen() }

    override suspend fun getJsonFromApi(): BreedResult =
        client.get<BreedResult> { url {
            takeFrom("https://dog.ceo/")
            encodedPath = "api/breeds/list/all"
        }
    }
}
\end{lstlisting}
