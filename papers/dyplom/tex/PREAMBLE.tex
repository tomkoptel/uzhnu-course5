\documentclass[a4paper,12pt]{report}
\usepackage[english, russian, ukrainian]{babel}
%%%%%%%%%%%%%%%%%%%%%%%%%%%% елеме
%\usepackage{psfig,graphicx,showkeys}
%%%%%%%%%%%%%%%%%%%%%%%%%%%%
%\usepackage[document]{ragged2e}
%    \usepackage{mathptmx}

\usepackage{amsmath,amssymb,euscript,amsthm}

% Пакет для включения рисунков
%--------------------------------------
\usepackage{graphicx}   
\graphicspath{ {./img/} }
\DeclareGraphicsExtensions{.jpg,.pdf,.png}
%--------------------------------------

%\usepackage{setspace}
\usepackage[nodisplayskipstretch]{setspace}

\usepackage{cite}
\usepackage{mathtools}
\usepackage{longtable}
\usepackage{color} %% это для отображения цвета в коде
\usepackage{listings} %% собственно, это и есть пакет listings

\usepackage{caption}
\DeclareCaptionFont{white}{\color{white}} %% это сделает текст заголовка белым
%% код ниже нарисует серую рамочку вокруг заголовка кода.
\DeclareCaptionFormat{listing}{\colorbox{white}{\parbox{\textwidth}{#1#2#3}}}
\captionsetup[lstlisting]{format=listing}

%\usepackage{makeindx}
%\makeindex
%%%%%%%%%%%%%%%%%%%%%%%%%%

\parindent=1.25cm
\textwidth=15,5cm
\textheight=22cm
\hoffset=-0.8cm
\voffset=-1.3cm

\pagestyle{myheadings}
\makeatletter
\@addtoreset{equation}{section}
\makeatother
\makeatletter
\@addtoreset{section}{part}
\makeatother
\makeatletter
\@addtoreset{thm}{section}
\makeatother
\makeatletter
\@addtoreset{lem}{section}
\makeatother
\makeatletter
\@addtoreset{expl}{section}
\makeatother
\makeatletter
\@addtoreset{remk}{section}
\makeatother
\makeatletter
\@addtoreset{nsl}{section}
\makeatother
\makeatletter
\@addtoreset{defn}{section}
\makeatother

\usepackage[center]{titlesec}
\titleformat{\chapter}[block]{\centering\fontseries{bx}\fontsize{14pt}{14pt}\selectfont}{\chaptertitlename~\thechapter.}{12pt}{}
\titleformat{\section}[block]{\fontsize{14pt}{14pt}\bfseries\filcenter}{\thesection}{1em}{}

\titlespacing*{\chapter}{0pt}{-30pt}{8pt}
%\titlespacing*{\section}{\parindent}{*4}{*4}
%\titlespacing*{\subsection}{\parindent}{*4}{*4}
\titlespacing\section{0pt}{12pt plus 4pt minus 2pt}{0pt plus 2pt minus 2pt}

\usepackage{fancyhdr}
    \pagestyle{fancy}
    \fancyhf{}
    \fancyhead[R]{\thepage}
    \fancyheadoffset{0mm}
    \fancyfootoffset{0mm}
    \setlength{\headheight}{17pt}
    \renewcommand{\headrulewidth}{0pt}
    \renewcommand{\footrulewidth}{0pt}
    \fancypagestyle{plain}{ 
        \fancyhf{}
        \rhead{\thepage}}
        
        \usepackage{indentfirst}


\renewcommand{\baselinestretch}{1.5}
\renewcommand{\theequation}{\arabic{chapter}.\arabic{section}.\arabic{equation}}
%%%%%%%%%%%%%%%%%%%%%%%%%%%%%%%%%%
%\newcommand{\reft}[1]{\,\;\ref{#1}}
\newcommand{\st}{\setminus}
\newcommand{\wt}{\widetilde}
\newcommand{\wh}{\widehat}
\newcommand{\ve}{\varepsilon}
\newcommand{\vf}{\varphi}
\newcommand{\vk}{\varkappa}

\newcommand{\pt}{\partial}
\newcommand{\cB}{{\mathcal B}}
\newcommand{\cG}{{\mathcal G}}
\newcommand{\cD}{{\mathcal D}}
\newcommand{\cL}{{\mathcal L}}
\newcommand{\cH}{{\mathcal H}}
\newcommand{\cP}{{\mathcal P}}
\newcommand{\cF}{{\mathcal F}}
\newcommand{\cA}{{\mathcal A}}
\newcommand{\cE}{{\mathcal E}}
\newcommand{\1}{1\!\!\,{\rm I}}
\newcommand{\mbR}{{\mathbb R}}
\newcommand{\mbP}{{\mathbb P}}
\newcommand{\mbE}{{\mathbb E}}
\newcommand{\mbN}{{\mathbb N}}
\newcommand{\mbZ}{{\mathbb Z}}



\newcommand{\ov}{\overline}
\renewcommand{\lg}{\langle}
\newcommand{\rg}{\rangle}


\newcommand{\mfF}{{\mathfrak F}}
\newcommand{\mfM}{{\mathfrak M}}
\newcommand{\mfX}{{\mathfrak X}}
\newcommand{\mfL}{{\mathfrak L}}
\newcommand{\mfG}{{\mathfrak G}}
\newcommand{\mfS}{{\mathfrak S}}

\newcommand{\const}{\mathop{\rm const}}
\newcommand{\Acirc}{\mathop{A}\limits^{ \ \circ}}
\newcommand{\Dcirc}{\mathop{\Delta}\limits^{ \circ}}


%%%%%%%%%%%%%%%%%%%%%%%%%%%

%\renewcommand{\refname}{\centerline{СПИСОК ВИКОРИСТАНИХ ДЖЕРЕЛ}}
%\numberwithin{equation}{section}
\theoremstyle{plain}
\newtheorem{thm}{Теорема}
[section]
\newtheorem{lem}{Лема}
[section]
\newtheorem{tv}{Твердження}[section]
\newtheorem{hip}{Гіпотеза}[section]
\newtheorem{corl}{Corollary}[section]
\theoremstyle{definition}
\newtheorem{defn}{Означення}
[section]
\newtheorem{pos}{Позначення}
[section]
\newtheorem{nsl}{Наслідок}
[section]
\newtheorem{expl}{Приклад}[section]
\newtheorem{ex}{Упражнение}
[section]

\theoremstyle{remark}
\newtheorem{remk}{Зауваження}[section]
\renewcommand{\proofname}{Доведення}
\renewcommand{\thepart}{\arabic{part}}
\renewcommand{\thechapter}{\arabic{chapter}}
\renewcommand{\chaptername}{РОЗДІЛ}

\DeclareMathOperator{\fsign}{Fsign}
\DeclareMathOperator{\supp}{supp}

%\renewcommand{\partname}{\hspace{5cm}Глава}
\renewcommand{\thesection}{\arabic{chapter}.\arabic{section}}
\renewcommand{\contentsname}{\centerline{Зміст}}

\usepackage{titletoc}%
\titlecontents{chapter}% <section-type>
  [0pt]% <left>
  {\bfseries}% <above-code>
  {\chaptername\ \thecontentslabel:\quad}% <numbered-entry-format>
  {}% <numberless-entry-format>
  {\hfill\contentspage}% <filler-page-format>

\def\thechapter{\Roman{chapter}}
% Оформление библиографии и подрисуночных записей через точку
\makeatletter
\renewcommand*{\@biblabel}[1]{\hfill#1.}
%\renewcommand*\l@section{\@dottedtocline{1}{1em}{1em}}
\renewcommand{\thefigure}{\thesection.\arabic{figure}}  % Формат рисунка секция.номер
\renewcommand{\thetable}{\thesection.\arabic{table}}    % Формат таблицы секция.номер
%\def\redeflsection{\def\l@section{\@dottedtocline{1}{0em}{10em}}}
\makeatother

\makeatletter
\renewenvironment{thebibliography}[1]
    {\section*{\refname}
        \list{\@biblabel{\@arabic\c@enumiv}}
           {\settowidth\labelwidth{\@biblabel{#1}}
            \leftmargin\labelsep
            \itemindent 16.7mm
            \@openbib@code
            \usecounter{enumiv}
            \let\p@enumiv\@empty
            \renewcommand\theenumiv{\@arabic\c@enumiv}
        }
        \setlength{\itemsep}{0pt}
    }
\makeatother

%Code-specific snippets
%--------------------------------------
\usepackage{listings}
\usepackage{xcolor}

\definecolor{codegreen}{rgb}{0,0.6,0}
\definecolor{codegray}{rgb}{0.5,0.5,0.5}
\definecolor{codepurple}{rgb}{0.58,0,0.82}

\lstdefinestyle{light}{
commentstyle=\color{codegreen},
keywordstyle=\color{magenta},
numberstyle=\tiny\color{codegray},
stringstyle=\color{codepurple},
basicstyle=\ttfamily\footnotesize,
breakatwhitespace=false,
breaklines=true,
captionpos=b,
keepspaces=true,
numbers=left,
numbersep=5pt,
showspaces=false,
showstringspaces=false,
showtabs=false,
tabsize=2
}

\lstset{style=light}
%--------------------------------------

% How to reduce line space (leading) within an enumerate environment?
% https://tex.stackexchange.com/questions/43743/how-to-reduce-line-space-leading-within-an-enumerate-environment
\usepackage{enumitem}
\setlist[enumerate]{itemsep=0mm}

% Fixes the numbering of the images
\renewcommand{\thefigure}{\arabic{chapter}.\arabic{figure}}
\renewcommand{\thetable}{\arabic{chapter}.\arabic{table}}
