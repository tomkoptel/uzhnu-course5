\newpage
\pagenumbering{arabic}
\setcounter{page}{3}
\addcontentsline{toc}{chapter}{\bf Вступ}

\chapter*{\centerline{\bf Вступ}}\label{vstyp}

Розробка додатків з використанням оригінального фреймворку під ОС Android та iOS з використанням IDE Android Studio та Xcode
домінують у галузі розробки мобільних додатків. Серед недоліків котрі, що можна віднести до так званої native("нативної")
розробки відносять такі фактори як:

\begin{itemize}
    \item потреба в супроводжені різних сирцевих баз коду для різних платформ (iOS та Android);
    \item утримання розробників під конкретну платформу вимагає більше грошових ресурсів від компанії;
    \item дуплікація зусиль в супроводжені сирцевого коду для обох платформ.
\end{itemize}

Для того щоби оптимізувати кошти супроводження та пришвидшити розробку додатків виникли багато платформні рішення, такі як React Native, Flutter, KMM.

React Native та Flutter - найкращі багато платформні фреймворки для створення мобільних додатків для iOS та Android.
Кілька років спільнота мобільних розробників порівнюють React Native та Flutter.

З випуском Kotlin 1.4 вересні 2020 на сцені з'явився новий фреймворк Kotlin Multiplatform(KMM), що намагається змагатися з React Native та Flutter.

\textbf{Метою} даної бакалаврської роботи є дослідження та порівняння кросс-платформених інструментів для розробки мобільних додатків.

Для досягнення цієї мети, було поставлено наступні \textbf{завдання}:
\begin{itemize}[noitemsep,partopsep=0pt,topsep=0pt,parsep=0pt]
    \item розробка мобільного додатку, що зображує лист порід собак з використанням всіх 3 фреймворків;
    \item оцінка ринку праці та порівняння трендів найму серед вакансій для трьох платформ;
    \item аналіз кожної з платформ на базі характеристик: система побудування, виклик до інтернету, робота з локальною БД, тестування, структура проєкту.
\end{itemize}

Дана бакалаврська робота складається із двох розділів. Перший розділ присвячений теоретичній частині.
Другий розділ присвячений опису архітектури додатків. Код, що використовується в описі знаходиться в додатку до роботи.
