\newpage
\pagenumbering{arabic}
\setcounter{page}{3}
\addcontentsline{toc}{chapter}{\bf Вступ}

\chapter*{\centerline{\bf Вступ}}\label{vstyp}

Розробка додатків з використанням оригінального фреймворку під ОС Android та iOS з використанням IDE Android Studio та Xcode
домінують у галузі розробки мобільних додатків. Нажальі серед недоліків котрі можна віднести до так званої native("нативної")
розробки відносять такі фактори як:

\begin{itemize}
    \item потреба в супроводжені різних сирцевих баз коду для різних платформ (iOS та Android);
    \item найм розробників під конкретну платформу дорогий і трудомісткий процес;
    \item витрати на розробку та обслуговування дуже високі за рахунок дуплікації зусиль в супроводжені.
\end{itemize}

Щоби розв'язати проблему було впровадженне міжплатформенні рішення для мобільних розробок, такі як React Native, Flutter, KMM.

React Native та Flutter - найкращі мобільні фреймворки для створення мобільних додатків для iOS та Android.
Вже кілька років спільноти розробків порівнюють ці дві платформи.

Нещодавно на сцені з'явився новий фреймворк Kotlin Multiplatform(KMM), що намагається змагатися з React Native та Flutter.

\textbf{Метою} даної бакалаврської роботи є дослідження та порівняння платфор розробки мобільних додатків.

Для досягнення цієї мети, було поставлено наступні \textbf{завдання}:
\begin{itemize}[noitemsep,partopsep=0pt,topsep=0pt,parsep=0pt]
    \item розробка мобільного додатку, що відображає лист порід собак з використанням всіх 3 фреймворків;
    \item оцінка ринку праці та порівняння трендів найму серед вакансій для трьох платформ;
    \item аналіз кожної з платформ на базі характеристик: система побудування, виклик до інтеренету, робота з локальною БД, тестування, структура проекту.
\end{itemize}

Дана бакалаврська робота складається із чьотирьох розділів.
