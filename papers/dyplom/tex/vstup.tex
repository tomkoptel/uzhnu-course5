\newpage
\pagenumbering{arabic}
\setcounter{page}{3}
\addcontentsline{toc}{chapter}{\bf Вступ}

\chapter*{\centerline{\bf Вступ}}\label{vstyp}

React Native та Flutter - найкращі мобільні фреймворки для створення мобільних додатків для iOS та Android.
Вже кілька років спільноти розробків порівнюють ці дві платформи.

Розробка додатків з використанням оригінального фреймворку під ОС Android та iOS з використанням Android Studio та Xcode домінували у галузі мобільної розробки, поки не виникали певні проблеми.
\begin{itemize}
    \item Потреба в різних базах кодів для різних платформ (iOS та Android)
    \item Найм розробників під конкретну платформу дорого справа
    \item Витрати на розробку та обслуговування дуже високі
\end{itemize}

Щоби розв'язати проблему було впровадженне міжплатформенне рішення для мобільних розробок, такі як React Native та Flutter.

\textbf{Метою} даної бакалаврської роботи є дослідження та порівняння платфор розробки мобільних додатків.

Для досягнення цієї мети, було поставлено наступні \textbf{завдання}:
\begin{itemize}[noitemsep,partopsep=0pt,topsep=0pt,parsep=0pt]
    \item ???
    \item ???
    \item ???
\end{itemize}

Дана бакалаврська робота складається із трьох розділів і додатку.
