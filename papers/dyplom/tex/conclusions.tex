\newpage
\addcontentsline{toc}{chapter}{\bf Висновки}

\chapter*{\centerline{\bf Висновки}}\label{vysnovky}

Flutter vs React Native - хто переможець?
Ми вже висвітлювали основні характеристики Flutter і те, чим він відрізняється від своїх інших крос-платформних систем, тому залишається останнє питання: чи насправді це майбутнє розвитку мобільних пристроїв?
Чи перемагає він у гонці React Native vs Flutter?

Хоча обидва фреймворки дійсно чудово підходять для розробки мобільних додатків, Flutter пропонує безліч функцій, які можуть допомогти нам розробляти красиві мобільні додатки з кращим користувацьким досвідом і робити це швидше - дозволяючи заощадити більше часу та грошей.
Flutter вже зарекомендував себе у світі розробки мобільних додатків, і я думаю, що 2021 рік - ідеальний час, щоб нарешті спробувати, якщо ви розглядаєте його для свого наступного мобільного MVP.

Якщо ви починаєте свій проект з нуля і збираєтесь поставити \textbf{POC (доказ концепції) або \textbf{MVP (мінімальний життєздатний продукт)Ї у найкоротший термін, подумайте про використання Flutter або React Native.
Але будьте готові зіткнутися з технологічними обмеженнями і завжди бути на кілька кроків позаду ринку, або знову переписувати мобільні програми, використовуючи рідні технології.
Якщо у вас є \textbf{"нативні" мобільні додатки} або ви хочете отримати максимум від власних мобільних платформ, розгляньте можливість використання \textbf{Kotlin Multiplatform}, яка вимагає майже нульових зусиль розробників Android, але може заощадити багато часу на стороні iOS.
У гіршому випадку ви просто можете переписати невдалу спільну частину на Swift.
Говорячи про час і витрати на розробку мобільних додатків, офіційної статистики щодо жодної з платформних платформ немає.
Але якщо ми вважаємо зусилля, необхідні для розробки двох окремих власних мобільних додатків, 100\%, будь-яке із згаданих у цій статті рішень може \textbf{заощадити близько 30\% здатності вашої команди} залежно від складності вашого проекту.
