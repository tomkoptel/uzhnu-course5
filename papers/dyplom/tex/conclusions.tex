\newpage
\addcontentsline{toc}{chapter}{\bf Висновки}

\chapter*{\centerline{\bf Висновки}}\label{vysnovky}

Flutter vs React Native vs Kotlin Multiplatform - хто переможець?

Якщо ви починаєте свій проект з нуля і збираєтесь поставити \textbf{POC (доказ концепції) або \textbf{MVP (мінімальний життєздатний продукт) у найкоротший термін, подумайте про використання Flutter або React Native.
Для чого не Kotlin Multiplatform? Справа в тому, що Kotlin Multiplatform це рішення, що фокусується на рішенні проблеми
спільної бізнес логіки. Розробка UI презентації додатку лягає на плечі native платформи і потребує навичок роботи з
відповідними інструментами Android/iOS стеків.

React Native та Flutter розв'язує проблему охоплюючи не тільки рівень бізнес логіки та роботи з даними з інернету та/або
локальної бази даних але й надає інструменти для розробки спільної UI презентації.

Серед недоліків розробки з React Native та Flutter - є відставання в доступному API.
Якщо додаток потребує низькорівневої інтеграції з системними інструментами, то використання Flutter та React Native
потребує реалізації рівнів адаптерів, що будуть надавати доступ до найновших можливостей платформи.

React Native це хороше ріщення для команди, що складається з Web розробників оскільки використовує React як рішення,
котре являється "де факто" стандартом в розробці веб сайтів. Отже має низький поріг входу.

В випадку Flutter, ми маємо дуже переспективний фреймворк, що активно розробляється та рекламується Google.
Розробка на Dart - це специфічна ніша і не є поширеною в громаді розробників, але не дивлячись на це
інфраструктура Flutter дозволяє в дуже короткі сроки розробити складний та красивий UI.
Це досягається за рахунок потужної та розжиренної системи віджетів. Недарма Flutter рекламує себе, як "все є віджет".

Kotlin Multiplatform виглядає, як меньш приваблива опція, але не давайте себе ввести в оману.
По факту код написаний з Kotlin Multiplatform дозволяє зібрати сирцевий код в рідному для цільової платформи
бінарному форматі, що дає нам високу ефективність під час виконання додатку.
По суті ми надалі працюємо з вихідними API платформ без додаткових абстракцій.

Враховуючи згадане можно зробити висновок, що React Native підходить для команд з досвідом веб-розробки.
Flutter потужний інструмент для швидкого старту проектів та бізнесів, що ще не здобули аудиторії.
Коли як Kotlin Multiplatform надає нам можливість надалі розробляти з використанням платформених інструментів,
поліпшуючи досвід розробки в перевикористанні бізнес логіки.
