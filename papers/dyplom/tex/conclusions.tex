\newpage
\addcontentsline{toc}{chapter}{\bf Висновки}

\chapter*{\centerline{\bf Висновки}}\label{vysnovky}

В даній роботі я спробував визначити найоптимальніший крос-платформений мобільний фреймворк поміж
Flutter, React Native та Kotlin Multiplatform.

Kotlin Multiplatform продемонстрував себе як рішення, що фокусується на рішенні проблеми
спільної бізнес логіки. Розробка UI презентації додатку лягає на плечі native платформи і потребує навичок роботи з
відповідними інструментами Android/iOS стеків.

React Native та Flutter розв'язує проблему, охоплюючи не тільки рівень бізнес логіки та роботи з даними з інернету та/або
локальної бази даних, але й надає інструменти для розробки спільної UI презентації.

Серед недоліків розробки з React Native та Flutter - є відставання в доступному API.
Якщо додаток потребує низькорівневої інтеграції з системними інструментами, то використання Flutter та React Native
потребує реалізації рівнів адаптерів, так званих "мостів"(bridges),
що будуть надавати доступ до найновших можливостей платформи.

React Native - це хороше ріщення для команди, що складається з Web розробників оскільки використовує React як рішення,
котре являється "де факто" стандартом в розробці веб сайтів. Отже має низький поріг входу для команд з досвідом розробки
web сторінок.

В випадку Flutter, ми маємо дуже переспективний фреймворк, що активно розробляється та рекламується Google.
Розробка на Dart - це специфічна ніша і не є поширеною в спільноті розробників, але не дивлячись на це
інфраструктура Flutter дозволяє в дуже короткі сроки розробити складний та красивий UI.
Це досягається за рахунок потужної та розжиренної системи віджетів. Недарма Flutter рекламує себе, як "все є віджет".

Kotlin Multiplatform виглядає, як меньш приваблива опція, але не давайте себе ввести в оману.
По факту код написаний з Kotlin Multiplatform дозволяє зібрати сирцевий код в рідному для цільової платформи
бінарному форматі, що дає нам високу ефективність під час виконання додатку. Звідси маємо найменьше використання CPU та
RAM в порівнянні Flutter та React Native, котрі заплатили вищим споживання ресурсів.
По суті, ми надалі працюємо з вихідним API платформ без додаткових абстракцій.

Майже в усіх трьох порівняльних тестах продуктивності React Native продемонструвала себе, як платформа з найбільшим
використанням RAM від 40\% до 140\% більше ніж Native та CPU від 380\% до 790\%. Flutter продемонстрував кращі результати
ніж React Native, а навіть перевершив нативну платформу в випадку холодного старту на 2 секунди для випадку тестування
продуктивності простих анімацій \ref{inVerita}.
Нажаль, при рендерінгу анімацій в обох тестах Flutter видав дуже низький FPS 8 та 9.
Найкращі результати в викорисатані ресурсів системи спостерігаються в нативних платформах, де RAM та CPU.

Враховуючи згадане можно зробити висновок, що React Native підходить для команд з досвідом веб-розробки.
Flutter потужний інструмент для швидкого старту проектів та бізнесів, що ще не здобули аудиторії.
Коли як Kotlin Multiplatform надає нам можливість надалі розробляти з використанням платформених інструментів,
поліпшуючи досвід розробки завдяки повторного використання бізнес логіки.
