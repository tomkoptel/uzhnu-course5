\pagenumbering{gobble}

\textbf{ПІБ}: Коптель Артем Олегович

\textbf{Назва}: React Native vs Flutter vs Kotlin Native. Порівняння кросс-платформених фрейморків для розробки мобільних додатків

\textbf{Факультет}: Інформаційних технологій

\textbf{Спеціальність}: 122 ``Комп'ютерні науки''

\textbf{Науковий керівник}: к. т. н., доц. О.В. Міца

По даній роботі опубліковано $0$ робіт.

\begin{center}
\textbf{АНОТАЦІЯ}
\end{center}

Дана робота присвячена дослідженню фреймворків розробки мобільних додатків,
порівнювання трендів серед вакансій, продуктивність платформ.
Ця тема є актуальною, тому що на даний час є необхідність в зменшення коштів на розробку та супроводження
мобільних додатків для Android та iOS платформ.

Бакалаврська робота складається з 2-ох розділів.

У першому розділі порівнюються тренди пошуків серед Flutter, React Native та Kotlin Multiplatform.
Далі йде теоретичний опис розробки на кожній з платформ з аналізом шарів абстракції для реалізації
контролю даних та презентації. Також у першому розділі проаналізовано тест продуктивності платформ.

У другому розділі опис архітектури додатків написаних під платформи Android/iOS з використанням
Flutter, React Native, Kotlin Multiplatform. Ідея додатків це зображення списку порід собак з
підтримкою режиму офлайн.

{\bf Ключові слова:} Flutter, React Native, Kotlin Multiplatform.

\newpage

\textbf{Name}: Artem Koptel

\textbf{Title}: React Native vs Flutter vs Kotlin Native. Comparison of cross-platform frameworks for the mobile development.

\textbf{Faculty}: Faculty of Information Technologies

\textbf{Speciality}: 122, Computer sciences

\textbf{Supervisor}: Dr. Alexander Mitsa. 

$0$ papers were submitted for publication on this topic.

\begin{center}
\textbf{ABSTRACT}
\end{center}

This work is dedicated to the study of frameworks for the development of mobile applications,
comparison of trends among vacancies, productivity of platforms.
This topic is relevant because there is currently a need to reduce funding for development and maintenance for the
mobile applications on both Android and iOS platforms.

The bachelor's thesis consists of 2 chapters.

The first section compares search trends among Flutter, React Native and Kotlin Multiplatform.
It contains a theoretical description of the development for each of the platforms with the analysis of abstraction data and presentation layers.
The first section also analyzes the platform performance test.

The second section describes the architecture of applications written for Android/iOS platforms using
Flutter, React Native and Kotlin Multiplatform. The idea of the app is to render a list of dog breeds with an
offline support.

{\bf Keywords:} Flutter, React Native, Kotlin Multiplatform.

