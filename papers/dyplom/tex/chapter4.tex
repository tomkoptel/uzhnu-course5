\chapter{Kotlin Multiplatform}
\label{ch4}


\section{UI та Kotlin Multiplatform}
\label{section.4.1}
Специфіка розробки з використанням Kotlin Multiplatform передбачує, що розробка рівня UI буде досягнута
інструментами нативними для платформи.

Для розробки UI під Android використовується декларативний XML, що відтворюється під час виконання в середовищі компоненту
контейнера так званої Activity(активність).
\begin{lstlisting}[style=light, language=Python,label={lst:android_xml},caption=Android UI with XML]
// 1
class MainActivity : AppCompatActivity() {
    override fun onCreate(savedInstanceState: Bundle?) {
        super.onCreate(savedInstanceState)
        // 2
        setContentView(R.layout.activity_main)
    }
}

// 3
<?xml version="1.0" encoding="utf-8"?>
<androidx.constraintlayout.widget.ConstraintLayout
    xmlns:android="http://schemas.android.com/apk/res/android"
    xmlns:app="http://schemas.android.com/apk/res-auto"
    xmlns:tools="http://schemas.android.com/tools"
    android:layout_width="match_parent"
    android:layout_height="match_parent"
    tools:context=".MainActivity"/>
\end{lstlisting}

\begin{enumerate}
    \item Декларуємо клас, що наслідує Activity.
    \item Конфігуруємо Activity посилаючись на XML, що визначає елементи UI.
    \item Безпосередньо сам XML, що декларує опис розмітки.
\end{enumerate}

Нещодавно з'явився тренд розробки UI з уживанням
Kotlin DSL(Domain Specific Language - Специфічна Мова Домену) так званий Jetpack Compose \cite{jetpack_compose}.

\begin{lstlisting}[style=light, language=Python,label={lst:android_jetpack_compose},caption=Android Jetpack Compose]
class MainActivity : AppCompatActivity() {
    override fun onCreate(savedInstanceState: Bundle?) {
        super.onCreate(savedInstanceState)

        val appContainer = (application as JetnewsApplication).container
        // 1
        setContent {
            // 2
            JetnewsApp(appContainer, navigationViewModel)
        }
    }
}
\end{lstlisting}

\begin{enumerate}
    \item Викликаємо функцію setContent, що приймає як аргумент функцію, що створює віджет.
    \item Створюємо віджет, який описує додаток.
\end{enumerate}

Як видно з прикладу \ref{android_jetpack_compose} Android екосистема взяла той самий керунок, що Flutter та React Native.
Такий самий тренд можна спостерігати і в розробці під iOS, де декларативний синтаксис описується з Swift UI \cite{swift_ui}.

\begin{lstlisting}[style=light, language=Python,label={lst:ios_swift_ui},caption=Swift UI]
import SwiftUI

// 1
@main
struct LandmarksApp: App {
    // 2
    var body: some Scene {
       // 3
        WindowGroup {
            ContentView()
        }
    }
}
\end{lstlisting}

\begin{enumerate}
    \item Декларуємо структуру, що є вхідною точкою для запуску додатку.
    \item Декларуємо сцену, що визначає екран користувача.
    \item В декларативний спосіб описуємо UI.
\end{enumerate}

Як видно з прикладів \ref{android_jetpack_compose} та \ref{ios_swift_ui} Android та iOS теж вибрали стежку декларативної розробки UI.
В свою чергу стає зрозуміло, що Kotlin Multiplatform не розв'язує проблему спільної бази коду для коду UI.
Це дуже важливий момент, оскільки весь UI треба дуплікувати під кожну платформу.
Якщо Kotlin Multiplatform не розв'язує проблему, то для чого його використовувати?
Задачу яку собі поставив Kotlin Multiplatform, це опис спільної логіки, тобто звернення до бази даних, файлової системи, інернету, тощо.


\section{Запити до мережі з Ktor}
\label{section.4.2}
На даний момент серед кроссплатформених бібліотек, що дозволяє нам моделювання інтернет викликів, найпоширенішою - є розв'язання Ktor \cite{ktor_home_page}.
В додатку коду, що йде нижче \ref{kmm_ktor} ми розглянемо приклад реалізації клієнта, що надалі створює з'єднання з API сервісом, що повертає нам список порід собак.

\begin{lstlisting}[style=light, language=Python,label={lst:kmm_ktor},caption=Ktor]
import io.ktor.client.HttpClient
import io.ktor.features.json.*
import io.ktor.request.*
import io.ktor.http.takeFrom

// 1
interface KtorApi {
    // 2
    suspend fun getJsonFromApi(): BreedResult
}

class DogApiImpl : KtorApi {
    // 3
    private val client = HttpClient {
        install(JsonFeature) { serializer = KotlinxSerializer() }
    }

    // 4
    init { ensureNeverFrozen() }

    // 5
    override suspend fun getJsonFromApi(): BreedResult =
        client.get<BreedResult> { url {
            takeFrom("https://dog.ceo/")
            encodedPath = "api/breeds/list/all"
        }
    }
}
\end{lstlisting}

\begin{enumerate}
    \begin{item}
        Визначаємо інтерфейс, що повертає нам результат виконання виклику до API.
    \end{item}

    \begin{item}
        Важливим момент є те, що функція визначена ключовим словом suspend.
        suspen маркерує нашу функцію, як ту, що буде виконувати "блокуючу" операцію.
        Виконання викликів до інтернету операція дорога і можевиконуватися секундами.
    \end{item}

    \begin{item}
        Створюємо об'єкт Ktor клієнта та конфігуруємо Json Serializer/Deserializer.
    \end{item}

    \begin{item}
        \textbf{ensureNeverFrozen} гарантує, що freezing(заморожування) закінчиться помилкою FreezingException.
        Дозволяє уникнути небажаного ефекту freezing, що є основою безпечного контролю посилань між паралельними потоками
        під час виконання програмою асинхроних викликів.
    \end{item}

    \begin{item}
        Реалізуємо функцію інтерфейсу декларуючи посилання, що нам поверне результат.
    \end{item}

\end{enumerate}

Як ми бачимо реалізація виклику за допомогою Ktor та Kotlin, хоча виглядає громоздко, але не є складною.
Більшість логіки, що створює сокет і опрацьовую потік байтів енкапсульвона за публічним інтерфейсом.
Від клієнта потребується декларування викликів та дизайн DTO(Data Transfer Object - Транспортний Об'єкт Даних),
в нащому випадку це клас, що описує структуру Json відповіді BreedResult.
