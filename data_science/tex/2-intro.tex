\Introduction

Ми почнемо з огляду того, як працюють моделі машинного навчання та як вони використовуються.
Це може здатися базовим, якщо ви раніше займалися статистичним моделюванням чи машинним навчанням.

В даній роботі опрацьовується наступний сценарій:

Ваш двоюрідний брат заробив мільйони доларів спекулюючи на нерухомості.
Він запропонував стати з вами діловими партнерами за умовою, що він надає фінансування, кол ви моделі.
Ці моделі мають передбачувати вартість будинків спираючись на дані.
Вартість має максимально точно оцінювати ціни на різні будинки.
Чим краще модель тим краща передбачувана вартість, і в результаті ближчи до реальності ціна на нерухомість.

Ви запитаєте свого кузена, як він прогнозував вартість нерухомості в минулому і він каже, що це просто інтуїція.
Але додаткові опитування виявляють, що він визначив закономірності цін на будинки, які він бачив у минулому, і він використовує ці моделі, щоб робити прогнози щодо нових будинків, які він розглядає.
Машинне навчання працює по такому же принципу.

Ми почнемо з моделі, яка називається Деревом рішень.
Є вигадливіші моделі, які дають точніші прогнози.
Але дерева прийняття рішень легко зрозуміти, і вони є базовим елементом для найкращих моделей в галузі даних.

Для простоти ми почнемо з максимально простого дерева рішень.
Далі ми розглянемо питання оцінки результатів на значення середньої абсолютної помилки \textbf(MAE).
Оцінивши похибку ми спробуємо оптимізувати модель.
На останок, ми розглянемо альтернативну модель "випадковий ліс дерев" та порівняємо її ефективність з "деревом рішень".
