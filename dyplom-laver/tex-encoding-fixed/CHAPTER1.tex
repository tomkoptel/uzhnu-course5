\newpage


\renewcommand{\proofname}{Доведення}
\renewcommand{\chaptername}{РОЗДІЛ}
\chapter{Основні алгебраїчні структури}
\label{ch1}

\section{Поняття групи}
\label{section.1.1}

В даному розділі наводяться основні означення із алгебри \cite{Lang2002}.

Нехай $M$ i $N$ -- непорожні множини. Множина усіх впорядкованих пар $(x,y)$ ($x \in M$, $y \in N$) називається декартовим добутком множин $M$ i $N$. Позначається цей добуток символом $N\times N$.

Бінарною алгебраїчною операцією на множині $M$ називається відображення $f$ множини $M \times M$ в множину $M$, $f: M \times M \to M$. Отже, бінарна алгебраїчна операція $f$ на множині $M$ -- це деяке правило, за яким кожній впорядкованій парі $(a,b)$ ($a,b \in M$) ставиться у відповідність певний елемент $f(a,b)$ з множини $M$. Позначатимемо $f(a,b)$ так:
\[
f(a,b) = a \cdot b = ab.
\] 
Такий запис операції називають мультиплікативним, $f$ -- множенням, а $ab$ -- добутком елементів $a$ i $b$.

\textbf{Групою} $\langle G, * \rangle$ називається множина $G$, на якій визначена бінарна операція $\cdot: G\times G \to G$, для якої виконуються наступні аксіоми:
\begin{enumerate}[noitemsep,partopsep=0pt,topsep=0pt,parsep=0pt]
\item $\forall a,b,c \in G \quad (a*b)*c=a*(b*c)$ (асоціативність);
\item $\exists e \in G: \forall a \in G \quad a*e=a$ (існування нейтрального елементу);
\item $\forall a \in G \; \exists (a^{-1}) \in G: \quad a*a^{-1}=e$ (існування симетричного елементу);
\end{enumerate}
Якщо крім цього для будь-яких двох елементів $a,b \in G$ виконується $a+b=b+a$ (комутативність), то група $G$ називається \textbf{комутативною} або \textbf{абелевою}.

Часто операцію в скінченній групі задаємо за допомогою \textbf{таблиці Келі}. Для цього, якщо група $G$ має скінченний порядок $n$, нумеруємо її елементи, починаючи з одиниці: $e$, $a_2$, $\ldots$, $a_n$. Утворюємо квадратну таблицю, в комірки якої записуємо результат операції $a_i * a_j$ для довільних двох елементів $a_i, a_j \in G$.

Слід зауважити, що зазвичай для позначення групової операції замість знака $*$ використовують більш звичні знаки $+$ та $\cdot$. Якщо беруть знак $+$, то алгебраїчну операцію називають додаванням, групу відносно цієї операції -- \textbf{адитивною}, нейтральний елемент -- \textbf{нульовим}, симетричний елемент -- \textbf{протилежним}. Якщо використовують знак $\cdot$, то алгебраїчну операцію називають множенням, групу відносно цієї операції -- \textbf{мультиплікативною}, нейтральний елемент -- \textbf{одиничним}, симетричний елемент -- \textbf{оберненим}. 

Наведемо деякі приклади груп:
\begin{enumerate}[noitemsep,partopsep=0pt,topsep=0pt,parsep=0pt]
\item $\langle \mathbb{Z}, + \rangle$, $\langle \mathbb{Q}, + \rangle$, $\langle \mathbb{R}, + \rangle$ -- абелеві групи. Ці групи називають адитивними групами відповідно цілих, раціональних та дійсних чисел.

\item Нехай $\mathbb{Q}^*=\mathbb{Q} \setminus \{0\}$, $\mathbb{R}^*=\mathbb{R} \setminus \{0\}$, $\mathbb{C}^*=\mathbb{C} \setminus \{0\}$. Групи $\langle \mathbb{Q}, \cdot \rangle$, $\langle \mathbb{R}, \cdot \rangle$, $\langle \mathbb{C}, \cdot \rangle$ також абелеві. Ці групи називають мультиплікативними групами відповідно цілих, раціональних та дійсних чисел.

\item Множина $O=\{ 0 \}$ відносно операції додавання утворює групу.

\item Множина $E=\{ 1 \}$ відносно операції множення утворює групу. Її називають одиничною групою.
\end{enumerate}



Нехай задано дві групи, між елементами якимх можна встановити взаємно однозначну відповідність. Якщо при цій відповідності результат алгебраїчної операції від довільних двох елементів однієї групи відповідає результату алгебраїчної операції від відповідних двох елементів іншої групи, то групи називається \textbf{ізоморфними}. Ізоморфні групи однаково влаштовані у смислі операції, тому в алгебрі їх не розрізняють, або вважають точними копіями одна одної. Таким чином, взаємно однозначне відображення $\phi$ групи $G$ на групу $G^*$ називається \textbf{ізоморфізмом}, якщо
\[
\phi (ab) = \phi(a) \phi(a) , (a,b \in G).
\]

Отже, дві групи $G$ та $G^*$ є ізоморфними, якщо існує ізоморфізм
\[
\phi: G \to G^*.
\]

Для будь-якої групи $G$ мають місце властивості (розглядаємо мультиплікативний запис):

\begin{enumerate} [noitemsep,partopsep=0pt,topsep=0pt,parsep=0pt]
\item $G$ має єдиний одиничний елемент;
\item для довільного елемента групи $G$ існує єдиний обернений елемент у цій групі;
\item $\forall a,b \in G$: $(ab)^{-1}=b^{-1} a^{-1}$;
\item $\forall a \in G$: $(a^{-1})^{-1} = a$.
\end{enumerate}

Нехай $G$ -- група, $a$ -- деякий елемент групи $G$. Елемент
\[
a^n = \underbrace{a\cdot \ldots \cdot a}_{\text{n разів}}, \qquad (n \in \mathbb{N})
\]
називається $n$-ою степінню елемента $a$. За означенням, 
\[
a^0 = e, \quad a^{-n} = (a^n)^{-1}, \quad (n \in \mathbb{N}).
\]

Якщо для довільного натурального числа $n$ $a^n \neq e$, то $a$ називається \textbf{елементом нескінченого порядку}.

Нехай $a$ -- елемент скінченного порядку, $n$ -- найменше з натуральних чисел $m$ таких, що $a^m = e$. Тоді число $n$ називають порядком елемента $a$.

Для довільних елементів $a,b \in G$ і цілих чисел $m$ та $n$ мають місце властивості:

\begin{enumerate}[noitemsep,partopsep=0pt,topsep=0pt,parsep=0pt]
\item $a^m a^n = a^{m+n}$;
\item $(a^m)^n = a^{mn}$;
\item $a^n = e$ тоді і тільки тоді, коли $n$ ділиться на порядок елемента $a$;
\item якщо елементи $a$ та $b$ комутують, тобто $ab=ba$, і їх порядки взаємно прості, то порядок елемента $ab$ рівний добутку порядків елементів $a$ та $b$.
\end{enumerate} 

Група, усі елементи якої, окрім одиничного, є нескінченного порядку, називається  \textbf{групою без скруту}. Група $G$ називається \textbf{періодичною}, якщо кожен її елемент є елементом скінченного порядку. Якщо порядки елементів групи обмежені в сукупності, то їх найменше спільне кратне називається \textbf{показником} (експонентою) групи.

Нехай $p$ -- просте число. Періодична група, порядок будь-якого елемента якої є степенем числа $p$, називається $p$-групою. Потужність множини $G$ називається \textbf{порядком} групи $G$ і позначається $|G|$. Якщо це число скінченне,то група $G$ називається \textbf{скінченною}, в іншому випадку група $G$ називається \textbf{нескінченною}. Група, яка складається із одного елемента, називається \textbf{тривіальною}.

Нехай $G$ -- група. Підмножина $H$ групи $G$ називається \textbf{підгрупою} групи $G$, якщо відносно алгебраїчної операції, заданої на $G$, $H$ є групою. Очевидно, $\{e\}$, $G$ є підгрупами групи $G$. Підгрупу групи $G$, відмінну від групи $G$, називають \textbf{власною} підгрупою групи $G$.

Для довільної групи $G$ виконуються наступні властивості:

\begin{enumerate}[noitemsep,partopsep=0pt,topsep=0pt,parsep=0pt]
\item Непорожня підмножина $H$ групи $G$ є підгрупою групи $G$ тоді і тільки тоді, коли для довільних елементів $a,b \in H$ виконуються умови:
\[
ab \in H, \qquad a^{-1} \in H.
\]
\item Добуток $A\cdot B = \{ ab | a \in A, b \in B \}$ підгруп $A$ та $B$ групи $G$ тоді і тільки тоді буде підгрупою групи $G$, коли $A\cdot B = B \cdot A$.

\end{enumerate}

Легко бачити, що перетин будь-якої непорожньої множини підгруп групи $G$ -- підгрупа групи $G$. Якщо $M$ -- довільна непорожня підмножина групи $G$, то перетин $\langle M \rangle$ усіх підгруп групи $G$, що містять $M$, називається підгрупою, \textbf{породженою} множиною $M$, а сама множина $M$ -- \textbf{системою твірних елементів} підгрупи $\langle M \rangle$. Підмножина $N$ множини $M$ називається \textbf{власною}, якщо $N \neq M$. Система твірних елементів деякої групи називається \textbf{незвідною}, якщо жодна її власна підмножина не є системою твірних елементів цієї групи.



\textbf{Циклічна група} -- це група, яка може бути породжена одним зі своїх елементів. Тобто усі елементи групи є степенями даного елемента:
\[
\langle a \rangle = \{a^n | n \in \mathbb{Z} \}.
\]

Формально, для мультиплікативних груп
\[
G = \langle a \rangle = \{ a^n \mid n \in \mathbb{Z} \},
\]
для адитивних:
\[
G = \langle a \rangle = \{ na \mid n \in \mathbb{Z} \}.
\]

Усі циклічні групи є абелевими. Прикладами циклічних груп є група цілих чисел $\mathbb{Z}$ з операцією додавання, група $\mathbb{Z} / n \mathbb{Z}$ цілих чисел за модулем $n$ з операцією додавання тощо.

Циклічна група має такі властивості:
\begin{enumerate}[noitemsep,partopsep=0pt,topsep=0pt,parsep=0pt]
\item Всяка нескінченна циклічна група $G$ ізоморфна адитивній групі цілих чисел $\mathbb{Z}^+$;
\item Всяка циклічна група порядку $n$ ізоморфна мультиплікативній групі $U_n$ комплексних коренів $n$-ого степеня з $1$. 
\item Всяка підгрупа циклічної групи -- циклічна.
\item Всяка підгрупа $H$ скінченної циклічної групи $G=\langle a \rangle$ порядку $n$ породжується елементом $a^s$, де $s$ -- дільник числа $n$, і є циклічною групою порядку $t$, причому $n = st$.
\item У циклічної групи скінченного порядку $n$ є стільки підгруп, скільки є дільників у числа $n$.
\end{enumerate}




\section{Діедральна група}\label{section.1.2}
Нехай $P_n$ -- опуклий $n$-кутник. Можемо розглядати його як нескінченно тонке тіло із сторонами, що є $n$-кутниками. Тіло із двома сторонами називається діедроном (слово походить із грецької мови). Прикладом може слугувати звичайна монета. Група симетрій діедрона називається \textbf{дієдральною групою}.

\begin{figure}[h!]
  \includegraphics[width=\linewidth]{dihedral.png}
  \caption{Приклад симетрій восьмикутника}
  \label{fig:dih}
\end{figure}

Покладемо центр симетрій діедрона у початок координат і розглянемо усі можливі перевторення, які відображають діедрон сам у себе. Ці перетворення включають обертання
\[
1, a, a^2, \ldots, a^{n-1}
\]
на $\tfrac{360}{n} \cdot k$ ($k=0,1,\ldots,n-1)$ градусів відносно осі $l$, яка є перепендикулярною до площини $P_n$ і проходить через початок координат. Також розглянемо обертання на $180 \cdot r$ ($r=1,2$) градусів навколо осі симетрії звичайного многокутника (див. рис. \ref{fig:dih}) . Позначимо через $b$ одне з таких відбиттів. Тоді група $D_{2n}$ складається із елементів
\[
1,a,a^2,\ldots,a^{n-1},b,ab,a^2 b, \ldots, a^{n-1} b
\] 
і відношень $a^n = 1$, $b^2 = 1$, $a b = b a^{-1}$.

У випадку $n=2$ отримуємо так-звану групу Клейна, таблицю Келі для якої наведено у таблиці \ref{tabklein}.

Отже, діедральна група може бути записана у вигляді:
\[
D_{2n} = \{ b| a^n =1, b^2 = 1, ab=ba^{-1}\}.
\]

Тут $\{a,b\}$ є генераторами групи $D_{2n}$ і є три співвідношення між генераторами.



\begin{table}[h!]
 \centering 
 \normalsize 
\begin{tabular}{|l|l|l|l|l|}
\hline
$\cdot$ & $1$ & $a$ & $b$ & $ab$ \\ \hline
$1$ & $1$ & $a$ & $b$ & $ab$ \\ \hline
$a$ & $a$ & $1$ & $ab$ & $b$ \\ \hline
$b$ & $b$ & $ab$ & $1$ & $a$ \\ \hline
$ab$ & $ab$ & $b$ & $a$ & $1$ \\ \hline
\end{tabular}
\caption{Таблиця Келі для групи Клейна}
\label{tabklein}
\end{table} 



\section{Група кватерніонів}\label{section.1.3}
Розглянемо наступну множину комплексних матриць:
\[
\begin{split}
& \pm \begin{pmatrix}1&0\\0&1\end{pmatrix},
 \pm \begin{pmatrix}i&0\\0&-i\end{pmatrix}, \\
& \pm \begin{pmatrix}0&1\\-1&0\end{pmatrix},
 \pm \begin{pmatrix}0&i\\i&0\end{pmatrix}
\end{split}  
\]

Дана множина матриць є групою восьмого порядку відносно операції множення, і називається \textbf{групою кватерніонів}. Дана група позначається $Q$.

Позначимо 
\[ 
a= \pm \begin{pmatrix}i&0\\0&-i\end{pmatrix}, \qquad b= 
 \pm \begin{pmatrix}0&1\\-1&0\end{pmatrix}. 
 \]
 
  Тоді
 \[
 Q = \{ 1,a,a^2,a^3, b, ab,a^2b,a^3b\}.
 \]

Тоді
\[
Q = \langle a,b | a^4 = 1, b^2 = a^2, ab=ba^{-1} \rangle.
\]

Таблиця  є таблицею Келі для групи кватерніонів порядку 8.


\begin{table}[h!]
 \centering
\normalsize 
\begin{tabular}{|l|l|l|l|l|l|l|l|l|}
\hline
$\cdot$ & $1$ & $a$ & $a^2$ & $a^3$ & $b$ & $ab$ & $a^2b$ & $a^3b$ \\ \hline
$1$ & $1$ & $a$ & $a^2$ & $a^3$ & $b$ & $ab$ & $a^2b$ & $a^3b$ \\ \hline
$a$ & $a$ & $a^2$ & $a^3$ & $1$ & $ab$ & $a^2b$ & $a^3b$ & $b$ \\ \hline
$a^2$ & $a^2$ & $a^3$ & $1$ & $a$ & $a^2b$ & $a^3b$ & $b$ & $ab$ \\ \hline
$a^3$ & $a^3$ & $1$ & $a$ & $a^2$ & $a^3b$ & $b$ & $ab$ & $a^2b$ \\ \hline
$b$ & $b$ & $a^3b$ & $a^2b$ & $ab$ & $a^2$ & $a$ & $1$ & $a^3$ \\ \hline
$ab$ & $ab$ & $b$ & $a^3b$ & $a^2b$ & $a^3$ & $a^2$ & $a$ & $1$ \\ \hline
$a^2b$ & $a^2b$ & $ab$ & $b$ & $a^3b$ & $1$ & $a^3$ & $a^2$ & $a$ \\ \hline
$a^3b$ & $a^3b$ & $a^2b$ & $ab$ & $b$ & $a$ & $1$ & $a^3$ & $a^2$ \\ \hline
\end{tabular}
\caption{Таблиця Келі для групи кватерніонів}
\label{tabq}

\end{table}


\section{Кільця та поля}\label{section.1.4}


\textbf{Кільцем} $\langle R, +, \cdot \rangle$ називається множина $R$, на якій визначені дві операції, $+: R \times R \to R$ та $\cdot: R \times R \to R$, для яких виконуються наступні аксіоми:
\begin{enumerate}[noitemsep,partopsep=0pt,topsep=0pt,parsep=0pt]
\item $\forall a,b \in R \quad a+b=b+a$ (комутативність додавання);
\item $\forall a,b,c \in R \quad (a+b)+c=a+(b+c)$ (асоціативність додавання);
\item $\exists \mathbf{0} \in R: \forall a \in R \quad a+\mathbf{0}=a$ (існування нульового елементу);
\item $\forall a \in R \; \exists (-a) \in R: \quad a+(-a)=0$ (існування оберненого елементу відносно додавання);
\item $\forall a,b,c \in \mathbb{F} \quad (a \cdot b) \cdot c=a \cdot (b \cdot c)$ (асоціативність множення);
\item $\forall a,b,c \in \mathbb{F} \quad (a+b) \cdot c=(a \cdot c)+(b \cdot c)$, $\quad   c \cdot (a+b)=c \cdot a+c \cdot b$ (дистрибутивність додавання відносно множення).
\end{enumerate} 




Якщо у кільці міститься одиничний елемент відносно множення, то таке кільце називається кільцем з одиницею. Якщо множення є комутативним, то таке кільце називається комутативним.

Прикладом кілець є зокрема цілі числа $\mathbb{Z}$ зі звичайним додаванням та множенням, що  утворюють комутативне кільце. 


\textbf{Полем} $\langle \mathbb{F}, +, \cdot \rangle$ називається множина $\mathbb{F}$, на якій визначені дві операції, $+: \mathbb{F} \times \mathbb{F} \to \mathbb{F}$ та $\cdot: \mathbb{F} \times \mathbb{F} \to \mathbb{F}$, для яких виконуються наступні аксіоми:
\begin{enumerate}[noitemsep,partopsep=0pt,topsep=0pt,parsep=0pt]
\item $\forall a,b \in \mathbb{F} \quad a+b=b+a$ (комутативність додавання);
\item $\forall a,b,c \in \mathbb{F} \quad (a+b)+c=a+(b+c)$ (асоціативність додавання);
\item $\exists \mathbf{0} \in \mathbb{F}: \forall a \in \mathbb{F} \quad a+\mathbf{0}=a$ (існування нульового елементу);
\item $\forall a \in \mathbb{F} \; \exists (-a) \in \mathbb{F}: \quad a+(-a)=0$ (існування оберненого елементу відносно додавання);
\item $\forall a,b \in \mathbb{F} \quad a \cdot b=b \cdot a$ (комутативність множення);
\item $\forall a,b,c \in \mathbb{F} \quad (a \cdot b) \cdot c=a \cdot (b \cdot c)$ (асоціативність множення);
\item $\exists e \in \mathbb{F}: \forall a \in \mathbb{F} \quad a \cdot e=a$ (існування одиничного елементу);
\item $\forall a \in \mathbb{F}, a \neq \mathbf{0} \; \exists (a^{-1}) \in \mathbb{F}: \quad a*a^{-1}=e$ (існування оберненого елементу відносно множення для ненульових елементів);
\item $\forall a,b,c \in \mathbb{F} \quad (a+b) \cdot c=(a \cdot c)+(b \cdot c)$ (дистрибутивність додавання відносно множення).
\end{enumerate} 


\textbf{Характеристикою поля} називаються найменше ціле додатне $n$, для якого виконується 
\[
n*e = \underbrace{e+\cdots+e}_{\text{$n$ доданків.}} = \textbf{0}.
\]
Якщо таке $n$ не існує, то характеристика поля приймається рівною нулю.


Полями є множини раціональних чисел $\mathbb{Q}$, дійсних чисел $\mathbb{R}$, комплексних чисел $\mathbb{С}$ зі звичними операціями додавання, віднімання, множення та ділення. Кожне із цих полів є розширенням попереднього, тобто $\mathbb{Q} \subset \mathbb{R} \subset \mathbb{C}$.
 


Нехай задано дві групи: $\langle G, * \rangle$ та $\langle H, \cdot \rangle$. \textbf{Гомоморфізмом} з $\langle G, * \rangle$ в $\langle H, \cdot \rangle$ називається функція $h: G \to H$ така, що для усіх $u, v \in G$ має місце
\[
h(u * v) = h(u) \cdot h(v),
\]
де групова операція зліва від знаку ``$=$'' відноситься до групи $G$, а операція по правий бік -- до групи $H$. 

Нехай $G$ -- група. \textbf{Характером} називається гомоморфізм із групи $G$ у мультиплікативну групу поля (зазвичай поля комплексних чисел).


Поля зі скінченною кількістю елементів називають \textbf{полями Галуа} і позначають $GF(q)$. . Будь яке скінченне поле $\mathbf{K}$ має просту характеристику $p>0$, тому воно містить в собі просте підполе $\mathbb{F}_p$. Довільний елемент поля $\mathbf{K}$ задається своїми $n$ координатами відносно певного базису, елементии якого належать $\mathbb{F}_p$. Таким чином, поле $\textbf{K}$ складається з $q=p^n$ елементів. І навпаки, для даних простого $p$ і натурального $n \geq 1$, існує єдине, не враховуючи автоморфізмів, поле Галуа з $q=p^n$ елементів, яке має характеристику $p$ і позначається $GF(q)= \mathbb{F}_q=\mathbb{F}_{p^n}$.

Ненульові елементи поля $\mathbb{F}_q$ утворюють групу щодо операції множення, яка називається \textbf{мультиплікативною групою поля} і позначається $\mathbb{F}_q^*$. Ця група є циклічною, тобто усі елементи отримуються піднесенням до степеня породжуючого.

Породжуючий елемент $\mathbb{F}_q^*$ називається також \textbf{примітивним елементом поля} $\mathbb{F}_q$. Поле  $\mathbb{F}_q$ містить $\varphi (q-1)$ примімтивних елементів, де $\varphi$ - функція Ейлера, що показує кількість натуральних чисел, менших за $q-1$ і взаємопростих із ним.

\textbf{Поліномом} (многочленом) однієї змінної над полем $GF(q)$ називається вираз вигляду
\[
f(x) = c_0 + c_1 x + \cdots + c_n x^n,
\] 
де коефіціенти $c_i \in GF(q)$, $\i \in \{0,\ldots,n\}$ є сталими коефіціентами, а $x$ - змінна. При цьому $n$ називається \textbf{степенем}  многочлена $f(x)$ і позначаєтсья $\deg (f)$.  Множину усіх многочленів над заданим полем $\mathbb{F}$ називають \textbf{кільцем многочленів} над $\mathbb{F}$ і позначають $\mathbb{F} [x]$. 


Нехай $\mathbb{F}$ деяке скінченне поле. Поліном $f(x) \in \mathbb{F}[x]$ називається \textbf{незвідним} у полі $\mathbb{F}$, якщо він не рівний константі і не дорівнює добутку двох або більше поліномів з $ \mathbb{F}[x]$, що не є константами.


Наймешим числом елементів поля є два ($0$ i $1$). Це поле позначають $GF(2)$. Додавання та множення у цьому полі виконуються за модулем два. Наведемо таблиці даних операцій.

\begin{table}[h!]
  \centering
  \normalsize 
\begin{tabular}{|c||c|c|}
\hline
+&0&1\\
\hline
\hline
0&0&1\\
\hline
1&1&0\\
\hline
\end{tabular}
\quad \quad \quad
\begin{tabular}{|c||c|c|}
\hline
*&0&1\\
\hline
\hline
0&0&0\\
\hline
1&0&1\\
\hline
\end{tabular}

\caption{Таблиці додавання та множення для $GF(2)$}

\end{table}
\section{Формування елементів полей Галуа}
\label{section.1.5}

Для заданого $q=p^n$, де $p$ - просте число, $n>1$, поле $GF(q)$ будується як фактор-кільце $\mathbb{F}_p[x] / (f(x))$, де $f(x) \in \mathbb{F}_p[x]$ є незвідним многочленом.

Розглянемо приклад. Для побудови поля Галуа $GF(2^4)$ слід вибрати незвідний многочлен четвертого степеня. Нехай $f(x)=x^4+x+1$. Кожен елемент поля матиме кілька форм представлення: показникову (як степінь примітивного елемента), поліноміальну (у вигляді відповідного многочлена), двійкову і десяткову.

Примітивний елемент $\alpha$ поля $GF(2^4)$ є коренем полінома $f(x)$, тож $f(\alpha)=0$. Маємо:  
\[
\alpha^4+\alpha+1=0,
\]
\[
\alpha^4=-1-\alpha.
\]
Оскільки ми маємо справу із полем характеристики $2$, додавання та віднімання є еквівалентними, тож можемо записати $\alpha^4=1+\alpha$. Користуючись цим записом, ми можемо представити всі вищі степені $\alpha$:
\begin{multline*}
\alpha^5=\alpha * \alpha^4 = \alpha (\alpha+1) = \alpha^2 + \alpha; \\
\alpha^6=\alpha * \alpha^5 = \alpha (\alpha^2+\alpha) = \alpha^3 + \alpha^2;\\
\alpha^7=\alpha * \alpha^6 = \alpha (\alpha^3+\alpha^2) = \alpha^3 + \alpha + 1 \ldots
\end{multline*}

Повний перелік всіх елементів $GF(2^4)$ наведено в таблиці \ref{tab1}.

\begin{table}[h!]   
  \centering
  \begin{tabular}{|c|c|c|c|}
\hline
Степінь & Многочлен & Двійкова форма & Десяткова форма\\
\hline
\hline
-- & 0 & 0000 & 0\\
\hline
$\alpha^0$ & 1 & 0001 & 1\\
\hline
$\alpha^1$ & $x$ & 0010 & 2\\
\hline
$\alpha^2$ & $x^2$ & 0100 & 4\\
\hline
$\alpha^3$ & $x^3$ & 1000 & 8\\
\hline
$\alpha^4$ & $x+1$ & 0011 & 3\\
\hline
$\alpha^5$ & $x^2+x$ & 0110 & 6\\
\hline
$\alpha^6$ & $x^3+x^2$ & 1100 & 12\\
\hline
$\alpha^7$ & $x^3+x+1$ & 1011 & 11\\
\hline
$\alpha^8$ & $x^2+1$ & 0101 & 5\\
\hline
$\alpha^9$ & $x^3+x$ & 1010 & 10\\
\hline
$\alpha^{10}$ & $x^2+x+1$ & 0111 & 7\\
\hline
$\alpha^{11}$ & $x^3+x^2+x$ & 1110 & 14\\
\hline
$\alpha^{12}$ & $x^3+x^2+x+1$ & 1111 & 15\\
\hline
$\alpha^{13}$ & $x^3+x^2+1$ & 1101 & 13\\
\hline
$\alpha^{14}$ & $x^3+1$ & 1001 & 9\\
\hline
\end{tabular}
  \caption{Елементи $GF(2^4)$}
\label{tab1}
\end{table}

\section{Групові кільця та групові алгебри}

Нехай $G$ -- мультиплікативна група з нейтральним елементом $e$, $K$ -- асоціативне кільце з одиницею $1$. Нехай $f: G \to K$ -- відображення, задане на групі $G$. Позначимо носій даного відображення через
\[
\supp (f) = \{ g \in G | f(g) \neq 0 \}.
\] 

Нехай
\[
KG = \{ u | u: G \to K, \supp(u) \text{ скінченний} \}.
\]

Кажемо, що $u,v \in KG$ є рівними, якщо $v(g)=u(g)$ для усіх $g \in G$.

Суму та добуток елементів $u$ та $v$ визначимо наступним чином:
\[
\begin{split}
(u+v)(g) &= u(g) + v(g)\\
(uv)(g) &= \sum_{h \in G} u(h) v(h^{-1} g).
\end{split}
\]

Очевидно, що множина $KG$ із таким чином заданими операціями додавання та множення утворює кільце. Дійсно,
\begin{itemize}[noitemsep,partopsep=0pt,topsep=0pt,parsep=0pt]
\item $KG$ є абелевою групою відносно додавання;
\item $KG$ є напівгрупою відносно множення;
\item дистрибутивні властивості виконуються:
\[
u(v+w) = uv+uw, \quad (v+w)u=vu+wu, \quad \forall u,v,w \in KG.
\]
\end{itemize}

Дане кільце називається \textbf{груповим кільцем}, а якщо $K$ є полем, то \textbf{груповою алгеброю}, і позначається $KG$. Елементи групового кільця можна представити у вигляді формальних сум наступного вигляду:
\[
\sum_{g \in G} u_g g.
\]
Нехай
\[
x = \sum_{g \in G} a_g g, \quad y = \sum_{g \in G} b_g g.
\]
Тоді
\[
\begin{split}
x + y &= \sum_{g \in G} (a_g + b_g) g, \\
xy &= \sum_{g \in G} \Big( \sum_{h \in G} a_h b_{h^{-1}g}\Big)g.
\end{split}
\]
Нехай задано $a \in K$, $x = \sum_{g \in G} с_g g \in KG$. Тоді
\[
a \cdot x = \sum_{g \in G} (a c_g)g, \quad x \cdot a = \sum_{g \in G} (c_g a)g.
\]


Розглянемо приклад. Нехай $G=D_8 = \langle a,b| a^4 = 1, b^2 = 1, bab = a^{-1} \rangle$ -- діедральна група восьмого порядку і $\mathbb{Z}$ -- кільце цілих чисел. Нехай $x = a+b-7ab$, $y=5+a^2-b-6a^2 b$, $x,y \in \mathbb{Z}G$. Тоді
\[
\begin{split}
x+y  & = a+b-7ab + 5+a^2-b-6a^2 b \\
     & = 5+a+a^2-7ab-6a^2b, \\
xy &= (a+b-7ab) \cdot (5+a^2-b-6a^2 b) = \\
& = -1+12a -6a^2 + 43a^3+5b-36ab+a^2b-13a^3b.
\end{split}
\]


Нехай $x = \sum_{g \in G} a_g g$ -- елемент $KG$, відмінний від нуля. Підгрупу
\[
\supp(x) = \{ g \in G | a_g \neq 0 \}
\]
групи $G$ називають \textbf{носієм елемента} $x$. Підгрупу, генеровану $\supp(x)$, називають \textbf{групою-носієм} елемента, позначаємо $\langle \supp(x) \rangle$.

При вивченні групових кілець важливим поняттям є також \textbf{довжина елемента}: якщо $x=0$, то довжина елемента рівна нулеві. У інших випадках довжина елемента рівна потужності носія.  
