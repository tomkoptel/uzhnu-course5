\pagenumbering{gobble}

\textbf{ПІБ}: Климчук Павло Володимирович

\textbf{Назва}: Програмна реалізація випадкових методів у системі комп'ютерної алгебри GAP System

\textbf{Факультет}: інформаційних технологій

\textbf{Спеціальність}: 122 ``Комп'ютерні науки''

\textbf{Науковий керівник}: кандидат фіз.-мат. наук Лавер В.О.

По даній роботі опубліковано $0$ робіт.

\begin{center}
\textbf{АНОТАЦІЯ}
\end{center}
%\chapter*{\Large \vspace*{-20mm}$$\mbox{АНОТАЦІЯ}$$}
%\addcontentsline{toc}{chapter}{Анотація} \thispagestyle{headings}


Дана робота присвячена випадковим методам у комп'ютерній алгебрі. Робота складається з трьох розділів. 

У першому розділі розглянуто загальні теоретичні відомості з алгебри, наведено означення групи, поля, групової алгебри, проілюстровано на прикладах побудову розширень скінченних полів. У другому розділі описано випадкові методи для дослідження певних властивостей алгебраїчних об'єктів. У третьому розділі описано практичну частину роботи. Наведено обґрунтування вибору середовища програмування, наведено інструкцію роботи користувача, роботу програми проілюстровано числовими прикладами.Програму написано на мові програмування високого рівня GAP.

Дана магістерська робота містить 3 розділи, 10 підрозділів.

{\bf Ключові слова:} скінченне поле, група, комп'ютерна алгебра, групова алгебра, випадкові методи.

\newpage

\textbf{Name}: Pavlo Klymchuk

\textbf{Title}: Program realization of random methods in computer algebra system GAP System

\textbf{Faculty}: Faculty of Information Technologies

\textbf{Speciality}: 122, Computer sciences

\textbf{Supervisor}: Dr. Vasyl Laver. 

$0$ papers were submitted for publication on this topic.

\begin{center}
\textbf{ABSTRACT}
\end{center}
%\chapter*{\Large \vspace*{-20mm}$$\mbox{АНОТАЦІЯ}$$}
%\addcontentsline{toc}{chapter}{Анотація} \thispagestyle{headings}

This bachelor's degree work contains three sections.
The first section is dedicated to the general concepts from algebra that are used in the sequel of the theses. The definitions of a group, a field, group algebra are given. Construction of extensions of a given finite field is illustrated by a numerical example. The second section contains information about random methods in computer algebra. The third section is devoted to the practical realization of the considered algorithms. The user's manual is given, the work of the program is illustrated by numerical examples. Program is written in GAP System -- an open source computer algebra system. 

Thesis work contains 3 chapters, 10 sections.

{\bf Keywords:} finite field, group, computer algebra, group algebra, random methods.

