\documentclass{article}
%Russian-specific packages
%--------------------------------------
\usepackage[T2A]{fontenc}
\usepackage[utf8]{inputenc}
\usepackage[ukrainian]{babel}
%--------------------------------------

%Hyphenation rules
%--------------------------------------
\usepackage{hyphenat}
\hyphenation{ма-те-ма-ти-ка вос-ста-нав-ли-вать}
%--------------------------------------

\title{Обробка зображень - Василь Олександрович Лавер}
\author{Aртем Коптель }
\date{Листопад 2020}

\begin{document}

    \maketitle
    \newpage

    \tableofcontents
    \newpage


    \section{Вступ}\label{sec:intro}

    Цей документ збирається представити деякі методи цифрової обробки зображень, засновані на просторовій обробці,
    які складають перетворення інтенсивності та просторову фільтрацію з використанням згладжуючих просторових фільтрів
    та посилення просторових фільтрів.
    Також будуть представлені результати цих методик.
    \newpage


    \section{Загaльні відомості}\label{sec:general}

    \subsection{Зображення}\label{subsec:image}
    Зображення, визначене з математичної точки зору, вважається функцією двох реальних змінних, наприклад,
    a (x, y) з а як амплітуда (наприклад, яскравість) зображення в реальній координатній позиції (x, у).
    Крім того, можна вважати, що зображення містить підзображення, які іноді називають регіонами інтересів,
    ROI або просто регіонами.
    Ця концепція відображає той факт, що зображення часто містять колекції предметів, кожен з яких може бути основою для регіону.

    \subsection{Цифрова обробка зображень}\label{subsec:image_preprocessing}
    Це використання комп’ютерних інструментів для виконання деяких процесів на цифровому зображенні, ці інструменти часто
    є комп’ютерним алгоритмом, що використовується для виконання певного завдання.
    Як підполе цифрової обробки сигналів, цифрова обробка зображень має багато переваг перед аналоговою обробкою зображень;
    це дозволяє застосовувати набагато ширший спектр алгоритмів до вхідних даних і дозволяє уникнути таких проблем,
    як накопичення шуму та спотворення сигналу під час обробки.

    \subsection{Процеси, які можна виконати при обробці зображень}\label{subsec:image_preprocessing_options}
    \begin{itemize}
        \item Такі геометричні перетворення, як збільшення, зменшення та обертання.
        \item Корекції кольорів, такі як регулювання яскравості та контрасту, квантування або Перетворення в інший колірний простір.
        \item Реєстрація (або вирівнювання) двох або більше зображень.
        \item Поєднання двох або більше зображень, напр. в середнє значення, суміш, різниця або зображення композитний.
        \item Інтерполяція та відновлення повного зображення із формату RAW.
        \item Сегментація зображення на регіони.
        \item Редагування зображень та цифрове ретушування.
        \item Розширення динамічного діапазону шляхом комбінування зображень з різним експозицією.
    \end{itemize}

    \subsection{Застосування обробки зображень}\label{subsec:image_preprocessing_application}
    \begin{itemize}
        \item Фотографія та друк
        \item Обробка супутникових зображень
        \item Медична обробка зображень
        \item Розпізнавання обличчя, Визначення особливостей, Ідентифікація обличчя
        \item Обробка зображень мікроскопом
    \end{itemize}


    \section{ПРОСТОРОВА ОБРОБКА ЗОБРАЖЕНЬ(SPATIAL PROCESSING)}\label{sec:spatial_processing}
    Термін \emph{Просторовий домен (Spatial domain)} відноситься до самої площини зображення, а методи обробки зображень засновані на безпосередній маніпуляції з пікселями на зображенні.
    Дві основні категорії просторової обробки включають перетворення інтенсивності та просторову фільтрацію.

    \emph{Трансформація інтенсивності(Intensity transformation)} діє на одиничні пікселі зображення для здійснення маніпуляцій з контрастом та порогового значення зображення.

    \emph{Просторова фільтрація(Spatial filtering)} стосується виконання таких операцій, як посилення зображення, працюючи в районі кожного пікселя на зображенні.
    Фільтр імені відноситься до прийняття або відхилення деяких компонентів, якщо взяти приклад частоти як компонент, що підлягає фільтруванню, фільтр, що пропускає низькі частоти, називається фільтром низьких частот.
    Ефект фільтра низьких частот полягає в розмитті (згладжуванні) зображення.
    Ми можемо здійснити подібне згладжування безпосередньо на самому зображенні, використовуючи просторові фільтри, які також називаються просторовими масками.

    \subsection{Згладжування просторових фільтрів (Smoothing spatial Filters)}\label{subsec:spatial_smoothing_filters}
    Згладжувальні фільтри використовуються для розмиття, а для зменшення шуму вихід лінійного згладжувального фільтра є середнім значенням пікселів, що містяться в районі маски фільтра.
    Ці фільтри іноді називають фільтрами усереднення.
    Це робиться шляхом заміни значення кожного пікселя на зображенні середнім рівнем інтенсивності в околиці, визначеному маскою фільтра, в результаті цього процесу виходить зображення із зменшеними різкими переходами в рівні інтенсивності.

    На малюнку вище показано два згладжувальні фільтри 3x3, у яких є просторовий фільтр усереднення
    коефіцієнти рівні, іноді називають коробчатим фільтром і коефіцієнтом, що має різні коефіцієнти виходу
    так зване середньозважене.
    Нижче наведено приклад гладкого зображення:

\end{document}
